In \citet{choi/etal:2016} a cross-correlation clustering analysis, between photometric redshift slices and overlapping spectroscopic redshifts, is used to determine linear biases in the CFHTLenS photometric redshift distributions.   The maximum redshift $z_B<0.9$ in this analysis was limited by the redshift overlap with the spectroscopic sample\footnote{The implication of a $z_B < 0.9$ high-redshift limit is that the analysis is unable to constrain the high-redshift tails of the redshift distributions due to the lack of overlapping high-redshift spectroscopic samples. As such the results from \citet{choi/etal:2016} are only the first important step towards a detailed understanding of the CFHTLenS tomographic redshift distributions and as such may be revised in the future.}.  \citet{choi/etal:2016} found significant biases in the redshift distributions used in the original CFHTLenS analyses,
concluding that any re-analysis of CFHTLenS should include systematic error terms to account for bias and scatter.    This re-analysis was presented in \citet{joudaki/etal:2016} where the impact of correcting for the biases determined by \citet{choi/etal:2016}, including their associated errors, served to reduce the overall constraining power of the survey and hence also the tension between constraints.  The best-fit model, however moved further from the Planck-preferred model, in agreement with the `toy-model' analysis of \citet{choi/etal:2016} which concluded that the effect of these biases would be to {\it reduce} the recovered amplitude of $\sigma_8$ by $\sim 4$\%. 

\citet{choi/etal:2016} additionally showed that applying a simple linear shift to the photometric redshift distributions was insufficient when modelling the true underlying redshift distribution.  In \citet{hildebrandt/etal:2016}, a direct calibration method was used to determine the redshift distribution of four tomographic bins.  Multiple bootstrap samples of the resulting calibrated distribution allowed for the characterisation of both the uncertainty in the mean and shape of the redshift distribution.  The inclusion of this more sophisticated treatment of photometric redshift errors did not alleviate the tension between Planck and KiDS parameter constraints.

\cite{kitching/etal:2016} use a linear relation fit to the biases determined by \citet{choi/etal:2016} in order to extrapolate the results beyond the photometric redshift limit of $z_B<0.9$ and determine a bias at each redshift.  They then correct the non-tomographic \citet{kilbinger/etal:2013} redshift distribution with this bias relation.  They report that this correction results in an {\it increase} to the recovered amplitude of $\sigma_8$ by $\sim 4$\%, which resolves the tension with Planck.   We are unable to recover this conclusion using two different methods.   Based on the \cite{kitching/etal:2016} relation, the predicted shift at the mean CFHTLenS redshift $z=0.747$ is $\Delta_z = 0.056$, where we use the convention from \citet{choi/etal:2016} that a positive $\Delta_z$ serves to increase the true mean redshift.  Using the same relation to individually shift each component of the full \citet{kilbinger/etal:2013} redshift histogram also results in a change to the mean redshift of $\Delta_z = 0.056$.  Increasing the mean redshift implies a reduction in the recovered amplitude of $\sigma_8$.  Using the same `toy-model' analysis of \citet{choi/etal:2016} we would expect this change in the mean redshift to {\it reduce} the recovered amplitude of $\sigma_8$ by $\sim 5$\%.    From this we suggest that \cite{kitching/etal:2016} could have misinterpreted the direction of the redshift biases reported in \citet{choi/etal:2016}. 



 