\documentclass[fleqn,usenatbib]{mnras} %
\pdfoutput=1

%% Packages
\usepackage{amsmath,amssymb,graphicx}
\usepackage{bm,color}% bold math
%\usepackage{ulem}
\usepackage{mathtools}
\usepackage{url}
\usepackage{footnote}
%\makesavenoteenv{tabular}
\makesavenoteenv{table}

\bibpunct{(}{)}{;}{a}{}{,}

\onecolumn

\pagestyle{empty}
\usepackage{psfrag, epsfig}


\newcommand{\ellbar}{\hbox{\it \l}\,}
\newcommand{\pref}{{\cal A}}
\newcommand{\edth}{\,\eth\,}
\renewcommand{\vec}{\bm}
\renewcommand{\d}[0]{{\rm d}}

\newcommand{\mk}[1]{{\bf\textcolor{blue}{#1}}}


\definecolor{purple}{RGB}{76, 0,153}
\newcommand{\ch}[1]{\textcolor{purple}{\bf #1}}


%%%%%%%%%%%%%%%%%%%%%%%%%%%%%%%%%%%%%%%%%%%%%%%%%%%%%%%%%%%%
\begin{document}
%%%%%%%%%%%%%%%%%%%%%%%%%%%%%%%%%%%%%%%%%%%%%%%%%%%%%%%%%%%%

\title[Precision calculations for cosmic shear]{Precision calculations for cosmic shear}  %??
%{Precision calculations of the cosmic shear power spectrum projection}

\author[M.~Kilbinger et~al.]
 {
  \parbox[h]{\textwidth}
  {
      Martin Kilbinger$^{1,2}$\thanks{E-mail: martin.kilbinger@cea.fr},
      %Catherine Heymans,
      %Peter Schneider,
      %Shahab Joudaki,
      %Marika Asgari,   % note that we'll worry about order at the end - Marika and Shahab put a lot of work into the comment paper, so we might just want to go alpha
      and other members of CFHTLenS and KiDS
      $\ldots$,
  }
  % 
  \vspace*{10pt} \\
  \hspace{-.1cm}$^1$ CEA/Irfu/SAp Saclay, Laboratoire AIM, 91191 Gif-sur-Yvette, France\\
  \hspace{-.1cm}$^2$ Institut d'Astrophysique de Paris, UMR7095 CNRS,
           Universit\'e Pierre \& Marie Curie, 98 bis boulevard Arago, 75014 Paris,
           France \\
 }

\voffset-0.50in


\date{\today}

\pagerange{\pageref{firstpage}--\pageref{lastpage}} \pubyear{2017}

\maketitle

\label{firstpage}


\begin{abstract}

We compute the spherical-sky weak-lensing power spectrum of the shear and
convergence. We discuss various approximations, such as flat-sky, first- and
second-order Limber equations for the projection. We find that the impact of adopting 
these approximations are negligible when constraining cosmological parameters from 
current weak lensing surveys.
This is demonstrated using data from the Canada-France-Hawaii Lensing Survey (CFHTLenS). 
We find that the reported tension with Planck CMB temperature anisotropy results cannot be alleviated, in
contrast to the recent claim made by \citet{2016arXiv161104954K}. 
For future large-scale surveys with unprecedented precision, we show that the spherical
second-order Limber approximation will provide sufficient accuracy, as it is in 
agreement with the full projection at the percent level for $\ell > 3$.  
In the spirit of reproducible research, our numerical implementation of all approximations
and the full projection are publicly available within the package
\textsc{nicaea} at \texttt{http://www.cosmostat.org/software/nicaea}.

\end{abstract}

\begin{keywords}
cosmological parameters -- methods: statistical
\end{keywords}

\input abbr-journals

%%%%%%%%%%%%%%%%%%%%%%%%%%%%%%%%%%%%%%%%%%%%%%%%%%%%%%%%%%%%
\section{Introduction}
\label{sec:intro}
%%%%%%%%%%%%%%%%%%%%%%%%%%%%%%%%%%%%%%%%%%%%%%%%%%%%%%%%%%%%

The measurement of weak gravitational lensing by large-scale structures
provides a powerful cosmological probe of dark matter, dark energy, and
modifications to gravity.  As such it is the primary science goal of several
current (KiDS, DES, HSC\footnote{KiDS: \href={http://kids.strw.leidenuniv.nl}, DES: \href={http://www.darkenergysurvey.org}, HSC: \href={http://hsc.mtk.nao.ac.jp/ssp/}})
and future (Euclid, LSST, WFIRST\footnote{Euclid: \href={http://sci.esa.int/euclid}, LSST: \href={http://www.lsst.org}, WFIRST: \href={http://wfirst.gsfc.nasa.gov}}) surveys.
%(Kilo-Degree Survey, KiDS \citet{kuijken/etal:2015}, Dark Energy Survey, DES, \citet{abbott/etal:2016}, Hyper-Suprime Camera Survey, HSC\footnote{HSC; \href={http://hsc.mtk.nao.ac.jp/ssp/}}) and future (Euclid, \citet{2011arXiv1110.3193L}, LSST, \citet{chang/etal:2013}, WFIRST\footnote{WFIRST; \href={http://wfirst.gsfc.nasa.gov}}) large-scale surveys.
Interest in the results from these surveys is high as statistically significant
deviations have been found between the cosmological parameter constraints from
the CMB Planck experiment \citep{2015arXiv150201589P} in comparison to weak
lensing constraints from both the Kilo-Degree Survey \citep[KiDS;][]{KiDS-450}
and the Canada-France-Hawaii Telescope Lensing Survey
\citep[CFHTLenS;][]{joudaki/etal:2016}.  If the source of this tension is not
a result of so-far unconsidered sources of systematic errors in one or all
experiments, extensions to the standard flat $\Lambda$CDM cosmological models
need to be considered. \citet{joudaki/etal:2017} have shown, for example, that
the tension can be resolved with an evolving dark energy model.

In the era of the upcoming large-scale surveys that will provide measurements of
cosmic shear with unprecedented precision, one needs to revisit the theoretical
predictions of the observables to ensure that the accuracy of the
models meet the accuracy of the upcoming data. In this paper, we examine the widely used Limber
approximation for the projected weak-lensing power spectrum. We consider
spherical coordinates and the flat-sky approximation, and compute the full
projection as well as first- and second-order Limber approximation. We show
that the second-order Limber approximation is an accurate representation of the full projection, with better than percent level precision for scale $\ell > 3$. 
Since this approximation involves only 1D integrals over the matter power spectrum, it is very fast to
calculate numerically and can readily employed in Monte-Carlo sampling methods
to obtain precision constraints on cosmological parameters.

This paper is organised as follows.  In section~\ref{sec:wl} we provide a pedagogical introduction to weak gravitational lensing theory, projections and power spectra for the flat-sky and spherical cases.  In section~\ref{sec:L2} we derive weak lensing observables using a second-order limber approximation first introduced by \citet{2008PhRvD..78l3506L}.  We compare the shear power spectrum and commonly-used two-point shear correlation function for the full solution to a range of different approximations in section~\ref{sec:comp}, providing cosmological constraints for each case using CFHTLenS data from \citet{CFHTLenS-2pt-notomo} in section~\ref{sec:cfhtlens}.  This paper draws from several sources of literature which have previously reviewed the accuracy of the Limber approximation for weak lensing studies, namely \citet{2008PhRvD..78d3002S,2012PhRvD..86b3001B, 2012MNRAS.422.2854G, 2016arXiv161104954K}.   We present a discussion and comparison of our results to these papers in Appendix~\ref{app:B}.


%%%%%%%%%%%%%%%%%%%%%%%%%%%%%%%%%%%%%%%%%%%%%%%%%%%%%%%%%%%%
\section{Weak-lensing projections and power spectra}
\label{sec:wl}
%%%%%%%%%%%%%%%%%%%%%%%%%%%%%%%%%%%%%%%%%%%%%%%%%%%%%%%%%%%%

In this section we review the basic weak-lensing projection expressions, and
compute lensing power spectra for a spherical case, and in the flat-sky
approximation. For completeness we provide a derivations of the weak lensing power spectra in Appendix~\ref{sec:derivations_C}.

%%%%%%%%%%%%%%%%%%%%%%%%%%%%%%%%%%%%%%%%%%%%%%%%%%%%%%%%%%%%
\subsection{The lensing potential}
\label{sec:psi}
%%%%%%%%%%%%%%%%%%%%%%%%%%%%%%%%%%%%%%%%%%%%%%%%%%%%%%%%%%%%


The lensing potential $\psi$ in the Born approximation is defined as the
projected 3D metric potential $\Phi$ along the line of sight
\citep{1998ApJ...498...26K,BS01},
%
% Hu00 (21,22) with minus sign
%
\begin{equation}
  \psi(\theta, \varphi) = \frac 2 {{\rm c}^2} \int_0^{\chi_{\rm lim}} \frac{{\rm d}\chi}{\chi}
    \Phi[\chi, f_K(\chi) \theta, f_K(\chi) \varphi; \chi] \, q(\chi),
  \label{eq:psi}
\end{equation}
%
where the lensing efficiency $q$ is given by
%
\begin{equation}
  q(\chi) = \int\limits_\chi^{\chi_{\rm lim}} {\rm d} \chi^\prime \, n(\chi^\prime)
    \frac{f_K(\chi^\prime - \chi)}{f_K(\chi^\prime)},
  %
  \label{eq:lens_efficiency}
\end{equation}
%
corresponding to a population of lensed galaxies with normalised source redshift
distribution $n(z) {\rm d}z = n(\chi) {\rm d} \chi$. Here, ${\rm c}$ is the
speed of light, the projection is carried out over comoving distances $\chi$ up
to a limiting distance $\chi_{\rm lim}$, and $f_K$ is the the comoving angular
distance, which depend on the spatial curvature of the Universe $K$.
%
This assumes a homogeneous galaxy distribution without clustering, so that the
redshift distribution in this approximation does not depend on the position
$(\theta, \varphi)$. Accounting for this position dependence leads to
correction of weak-lensing quantities due to clustering of source galaxies with
other sources \citep{2002A&A...389..729S}, and with galaxies associated with
lens structures \citep{1998A&A...338..375B,H02}.

The 3D potential is related to the density contrast $\delta$ via the Poisson
equation. Assuming General Relativity, this relation is written in Fourier space as
%
\begin{align}
  \hat \Phi(\vec k; \chi) = & - \frac 3 2 \Omega_{\rm m} H_0^2 k^{-2} a^{-1}(\chi) \hat \delta(\vec k; \chi),
      \label{eq:poisson}
\end{align}
%
where $\Omega_{\rm m}$ is the matter density parameter, $H_0$ the Hubble constant, $\vec k$ a 3D Fourier wave
vector, and $a$ the scale factor with $a=1$ today.
The Fourier transform of the potential and its inverse are defined as
%
\begin{align}
  \hat \Phi(\vec k; \chi) = &  \int {\rm d}^3 r \, \Phi(\vec r; \chi) {\rm e}^{{\rm i} \vec k \vec r} .
  \label{eq:hatPhi}
  \\
  \Phi(\vec r; \chi) = &  \int \frac{{\rm d}^3 k}{(2\pi)^3}
      \hat \Phi(\vec k; \chi) {\rm e}^{-{\rm i} \vec r \vec k},
  \label{eq:hatPhi_inv}
\end{align}
%
where the integration range for both integrals is $\mathbb{R}^3$.


%%%%%%%%%%%%%%%%%%%%%%%%%%%%%%%%%%%%%%%%%%%%%%%%%%%%%%%%%%%%
\subsection{Lensing power spectra in the spherical case}
%%%%%%%%%%%%%%%%%%%%%%%%%%%%%%%%%%%%%%%%%%%%%%%%%%%%%%%%%%%%

%%%%%%%%%%%%%%%%%%%%%%%%%%%%%%%%%%%%%%%%%%%%%%%%%%%%%%%%%%%%
\subsubsection{Lensing potential power spectrum}
%%%%%%%%%%%%%%%%%%%%%%%%%%%%%%%%%%%%%%%%%%%%%%%%%%%%%%%%%%%%

Following \cite{2000PhRvD..62d3007H} we decompose the lensing potential
(\ref{eq:psi}) into a spherical harmonics expansion, in analogy of CMB
temperature, both of which are scalar functions on the sphere. This
decomposition and its inverse are
%
% Hu00 (23)
%
\begin{align}
  \psi(\theta, \varphi) = \sum_{\ell=0}^\infty \sum_{m=-\ell}^\ell \psi_{\ell m} {\rm Y}_{\ell m}(\theta, \varphi);
    \label{eq:psi_harm_exp}
    \\
    %
  \psi_{\ell m} = \int_{\mathbb{S}^2} {\rm d} \Omega \, \psi(\theta, \varphi) {\rm Y}^\ast_{\ell m}(\theta, \varphi).
  \label{eq:psi_harm_exp_inv}
\end{align}

To specify tomographic redshift bins $i=0\ldots N_z-1$, we introduce a family
of lensing efficiency functions $q_i$ defined by a corresponding family of
redshift distributions $n_i$ via eq.~(\ref{eq:lens_efficiency}). The resulting
lensing potential is denoted by $\psi_{\ell m, i}$. The tomographic power
spectrum of the lensing potential between two redshift bins $i$ and $j$,
$C_{ij}^\psi$ is then defined by \cite{pee80}
%
\begin{equation}
  \left\langle \psi^{}_{\ell m, i} \, \psi^\ast_{\ell^\prime m^\prime, j} \right\rangle
    = \delta_{\ell \ell^\prime} \delta_{m m^\prime} C^\psi_{ij}(\ell) .
  \label{eq:C_ell_psi}
\end{equation}
%
Using the properties of the spherical harmonics (see
App.~\ref{sec:derivations_C} for details) the power spectrum can be written as
%
\begin{align}
  C_{ij}^\psi(\ell) = & \frac 8 {c^4 \pi} 
  \int_0^{\chi_{\rm lim}} \frac{{\rm d}\chi}{\chi} q_i(\chi)
  \int_0^{\chi^\prime_{\rm lim}} \frac{{\rm d}\chi^\prime}{\chi^\prime} q_j(\chi^\prime)
  \int {\rm d} k k^2 \, {\rm j}_\ell(k \chi) {\rm j}_\ell(k \chi^\prime) P_\Phi(k; \chi, \chi^\prime)
  \label{eq:C_ell_phi_Pphi} \\
  = & \frac 8 \pi \pref^2
  \int_0^{\chi_{\rm lim}} \frac{{\rm d}\chi}{\chi} \frac{q_i(\chi)}{a(\chi)}
  \int_0^{\chi^\prime_{\rm lim}} \frac{{\rm d}\chi^\prime}{\chi^\prime} \frac{q_j(\chi^\prime)}{a(\chi^\prime)}
  \int \frac{{\rm d} k}{k^2} \, {\rm j}_\ell(k \chi) {\rm j}_\ell(k \chi^\prime) P_{\rm m}(k; \chi, \chi^\prime) ;
  \label{eq:C_ell_phi_Pm}
\end{align}
%
where ${\rm j}_\ell$ it the spherical Bessel function of order $\ell$. For
convenience we defined the normalisation constant $\pref$ as
%
\begin{equation}
  \pref = \frac 3 2 \Omega_{\rm m} \left(\frac{H_0} c\right)^2.
  \label{eq:pref}
\end{equation}
%
The first line expresses $C_{ij}^\psi$ in terms of the 3D potential power spectrum $P_\Phi$, defined as
%
\begin{align}
  \left\langle \hat \Phi(\vec k; \chi) \hat \Phi^\ast(\vec k^\prime; \chi^\prime) \right\rangle
    = & (2\pi)^3 \delta_{\rm D}(\vec k - \vec k^\prime) P_\Phi(k; \chi, \chi^\prime).
    %\nonumber \\
    %= & (2\pi)^3 \delta_{\rm D}(\vec k - \vec k^\prime) \pref^2 k^{-4} a^{-1}(\chi) a^{-1}(\chi^\prime)
      %P_{\rm m}(k; \chi, \chi^\prime).
  \label{eq:p_phi}
\end{align}
%
The second line uses the 3D matter density power spectrum $P_{\rm m}$, which is defined analogously as
%
\begin{align}
  \left\langle \hat \delta(\vec k; \chi) \hat \delta^\ast(\vec k^\prime; \chi^\prime) \right\rangle
    = & (2\pi)^3 \delta_{\rm D}(\vec k - \vec k^\prime) P_{\rm m}(k; \chi, \chi^\prime),
  \label{eq:p_m}
\end{align}
%
and is
related to $P_\Phi$ via the absolute square of the Poisson equation (\ref{eq:poisson}).


In the following section we will discuss the relations between of shear and
convergence to the lensing potential on the sphere, and derive the power
spectrum of the former two fields.

%%%%%%%%%%%%%%%%%%%%%%%%%%%%%%%%%%%%%%%%%%%%%%%%%%%%%%
\subsubsection{Shear power spectrum}
%%%%%%%%%%%%%%%%%%%%%%%%%%%%%%%%%%%%%%%%%%%%%%%%%%%%%%

The shear $\gamma = \gamma_1 + {\rm i} \gamma_2$ is related to the potential at
linear order by the trace-free part of the Jacobi matrix. The involved
differential operator on the sphere is called \emph{edth} derivative, $\edth$,
see \cite{2005PhRvD..72b3516C} for a in-depth mathematical discussion of this
concept. The edth operator $\edth$ ($\edth^\ast$) raises (lowers) the spin $s$
of an object. Twice applying this operator to the scalar (spin-0) potential
creates the spin-2 shear:
%
\begin{align}
  \gamma(\theta, \varphi) = & \frac 1 2 \edth \edth \psi(\theta, \varphi).
    \nonumber \\
  \gamma^\ast(\theta, \varphi) = & \frac 1 2 \edth^\ast \edth^\ast \psi(\theta, \varphi).
  \label{gamma_psi_spher}
\end{align}
%
To write the shear on the sphere in terms of the lensing potential $\psi$, we
insert the harmonics expansion of the potential (\ref{eq:psi_harm_exp}). This
requires to compute second derivatives of the spherical harmonic functions.
This opereration defines a new object, the \emph{spin-weighted spherical
harmonic} $_s{\rm Y}_{\ell m}$. The shear can be written on the sphere in terms
of these functions as a spherical harmonics multipole expansion with
coefficients $_{\pm 2} \gamma_{\ell m}$. This expansion together with its inverse
is
%
% Hu04 (A10), van de Rijt PhD (6.32)
%
\begin{align}
  (\gamma_1 \pm {\rm i} \gamma_2)(\theta, \varphi) = & \sum_{\ell m} \,\, _{\pm 2}\gamma_{\ell m} \; _{\pm 2}\!{\rm Y}_{\ell m}(\theta, \varphi);
  \label{eq:gamma_harm_exp}
    \\
  \, _2 \gamma_{\ell m} = & \int_{\mathbb{S}^2} {\rm d} \Omega \, \gamma(\theta, \varphi) \;  _2\!{\rm Y}^\ast_{\ell m}(\theta, \varphi);
    \nonumber \\
  \, _{-2} \gamma_{\ell m} = & \int_{\mathbb{S}^2} {\rm d} \Omega \, \gamma^\ast(\theta, \varphi) \,  _{-2}\!{\rm Y}^\ast_{\ell m}(\theta, \varphi).
  \label{eq:gamma_harm_exp_inv}
\end{align}
%
The spin-weighted spherical harmonics $_s\!{\rm Y}_{\ell m}$ that are the basis function
in the expansion of the shear (\ref{eq:gamma_harm_exp}) can be calculated via the relations
%
% Rijt (6.16); BBRV12 (9)
%
\begin{align}
  \ellbar(\ell, s) \; _s\!{\rm Y}_{\ell m}(\theta, \varphi) = & \edth^s {\rm Y}_{\ell m}(\theta, \varphi);
    \nonumber \\
  \ellbar(\ell, s) \; _{-s}\!{\rm Y}_{\ell m}(\theta, \varphi) = & (-1)^s \left(\edth^\ast\right)^s {\rm Y}_{\ell m}(\theta, \varphi),
  \label{eq:sYlm_def} 
\end{align}
%
with the spin prefactor \citep{2012PhRvD..86b3001B}
%
\begin{equation}
  \ellbar(\ell, s) = \sqrt{\frac{(\ell + s)!}{(\ell - s)!}}.
  \label{eq:ellbar}
\end{equation} 
%
Inserting the lensing potential expansion (\ref{eq:psi_harm_exp}) into the
expression for the shear (\ref{gamma_psi_spher}), and using (\ref{eq:sYlm_def})
to compute the derivatives, we find for the shear expansion coefficients
\citep{2000PhRvD..62d3007H,2001astro.ph.11605T}
%
% Rijt after (7.5); Taylor01 (18); Hu00 (A10)
%
\begin{equation}
  _{\pm 2} \gamma_{\ell m} = \frac 1 2 \ellbar(\ell, 2) \psi_{\ell m}.
  \label{eq:gamma_ellm_phi_ellm}
\end{equation}
%
The two coefficients $_{+2} \gamma_{\ell m}$ and $_{-2} \gamma_{\ell m}$ are
identical since the potential $\phi$ is a real function.

The tomographic shear power spectrum, in analogy to (\ref{eq:C_ell_phi_Pm}), is defined by
%
\begin{equation}
  \left\langle _2\gamma^{}_{\ell m, i} \; {}_2\gamma^\ast_{\ell^\prime m^\prime, j} \right\rangle
    = \delta_{\ell \ell^\prime} \delta_{m m^\prime} C^\gamma_{ij}(\ell).
  \label{eq:C_ell_gamma}
\end{equation}
%
For a flat Universe, this is given by
%
\begin{align}
  C^\gamma_{ij}(\ell) = \frac 1 4 \ellbar^2(\ell, 2) \, C^\psi_{ij}(\ell)
                 = & \frac 2 \pi \, \ellbar^2(\ell, 2) \, \pref^2
                 \int_{0}^{\chi_{\rm lim}} \frac{\rm d \chi}{\chi} \frac{q_i(\chi)}{a(\chi)}
                \int_{0}^{\chi_{\rm lim}} \frac{\rm d \chi^\prime}{\chi^\prime}
                \frac{q_j(\chi^\prime)}{a(\chi^\prime)}
                %\nonumber \\
                %& \times
                \int_0^\infty \frac{{\rm d} k}{k^2} \, P_{\rm m}(k, \chi, \chi^\prime) \,
                {\rm j}_\ell(k \chi) \, {\rm j}_\ell(k \chi^\prime) .
  \label{eq:C_ell_full}
\end{align}
%
The spherical prefactor for the full projection power spectrum is $p(\ell) :=
\ellbar^2(\ell, 2)$, which will be modified under flat-sky and Limber
approximations below.

%%%%%%%%%%%%%%%%%%%%%%%%%%%%%%%%%%%%%%%%%%%%%%%%%%%%%%
\subsubsection{Convergence power spectrum}
%%%%%%%%%%%%%%%%%%%%%%%%%%%%%%%%%%%%%%%%%%%%%%%%%%%%%%

The convergence is related to the lensing potential on the sphere via the
product of spin-raising and spin-lowering edth operators, which are identical
to the spherical Laplacian differential operator.

%
\begin{equation}
  \kappa(\theta, \varphi) = \frac 1 2 \edth \edth^\ast \psi(\theta, \varphi) = \frac 1 2 \nabla^2 \psi(\theta, \varphi).
  \label{eq:kappa_psi_spher}
\end{equation}
%
The spherical harmonics are eigenfunctions of the Laplacian,
%
\begin{equation}
  \nabla^2 {\rm Y}_{\ell m}(\theta, \varphi) = - \ell (\ell + 1) {\rm Y}_{\ell m}(\theta, \varphi)
    = - \ellbar^2(\ell, 1) {\rm Y}_{\ell m}(\theta, \varphi),
  \label{eq:nabla_Ylm}
\end{equation}
%
The convergence power spectrum is then similar to the shear power spectrum
(\ref{eq:C_ell_gamma}) with a different spherical prefactor,
%
\begin{equation}
  C^\kappa_{ij}(\ell) = \frac 1 4 \ellbar^4(\ell, 1) \, C^\psi_{ij}(\ell)
    = \frac{\ell (\ell+1)}{(\ell-1)(\ell+2)} C^\gamma_{ij}(\ell) .
  \label{eq:C_ell_kappa_full}
\end{equation}
%
The convergence power spectrum is thus larger than the shear power spectrum, by
10\% for $\ell=4$, 1\% for $\ell = 14$, and less than 0.1\% for $\ell>45$.


%%%%%%%%%%%%%%%%%%%%%%%%%%%%%%%%%%%%%%%%%%%%%%%%%%%%%%
\subsection{Flat-sky approximation}
%%%%%%%%%%%%%%%%%%%%%%%%%%%%%%%%%%%%%%%%%%%%%%%%%%%%%%

Most previous work has used lensing quantities approximated on a flat sky,
neglecting the sky curvature. This is a valid approach for past and current
survey areas that have extents up to 10 degrees or less. The correlation
functions from the observed shear have been calculated using spherical
coordinates, and the effect on large scales is not negligible \cite{FSHK08}.
Here we examine the effect of sky curvature on the theoretical models.

For a flat sky, the spherical harmonics expansions are approximated by Fourier
transforms. The flat-sky equivalent of eqs.~(\ref{eq:psi_harm_exp}) and
(\ref{eq:psi_harm_exp_inv}) are
%
\begin{align}
  \psi(\vec \vartheta) = & \int \frac{{\rm d}^2 \ell}{(2\pi)^2} \, {\rm e}^{-{\rm i} \vec \ell \vec \vartheta} \hat \psi(\vec \ell);
  \label{eq:psi_fourier}
  \\
  %
  \hat \psi(\vec \ell) = & \int {\rm d}^2 \vartheta \, {\rm e}^{{\rm i} \vec \ell \vec \vartheta} \psi(\vec \vartheta);
  \label{eq:psi_fourier_inv}
\end{align}
%
where $\vec \vartheta = (\theta, \varphi)$ is the vector describing a 2D angle on the sky.
Instead of a harmonics coefficients $\psi_{\ell m}$, the Fourier representation of the potential
$\hat \psi$ now depends on the vector $\vec \ell \in R^2$.

The power spectrum, the flat-sky version of
(\ref{eq:C_ell_psi}) is defined by
%
\begin{equation}
  \left\langle \hat \psi_i^{}(\vec \ell) \hat \psi_j^\ast(\vec \ell^\prime) \right\rangle
    = (2\pi)^2 \delta_{\rm D}(\vec \ell - \vec \ell^\prime) P^\psi_{ij}(\ell).
  \label{eq:P_ell_psi}
\end{equation}


\mk{I can derive a flat-sky shear power spectrum by replacing $\ellbar^2(\ell, 2)$ with $\ell^4$
in (\ref{eq:C_ell_full}), equivalent to replacing the edth operators with Euclidian derivatives.
However, this results in a different expression compared to other references, e.g.~\cite{2008PhRvD..78d3002S},
which is the following equation.}

The flat-sky power spectrum is
%
\begin{align}
  P_{ij}^\gamma(\ell) = & \frac 2 \pi \, \pref^2
                 \int_{0}^{\chi_{\rm lim}} {\rm d} \chi \chi \, \frac{q_i(\chi)}{a(\chi)}
                \int_{0}^{\chi_{\rm lim}} {\rm d} \chi^\prime\, \chi^\prime
                \frac{q_j(\chi^\prime)}{a(\chi^\prime)}
                \nonumber \\
                & \times \int_0^\infty {\rm d} k \, k^2 \, P_{\rm m}(k; \chi, \chi^\prime) \,
                {\rm j}_\ell(k \chi) \, {\rm j}_\ell(k \chi^\prime) .
  \label{eq:P_ell_gamma}
\end{align}
%

\mk{I can get this by placing the prefactor under the integral and
setting $\ell^4 = k^4 \chi^2 {\chi^\prime}^2$. This does not seem to be the proper way.
Other derivations of the shear power spectrum (e.g.\cite{BS01} also assume other approximations
such as small-angle and Limber, but I do not know how to derive this without Limber.}


\section{Second-order Limber approximation for weak lensing}
\label{sec:L2}

\subsection{Spherical case}

We follow \cite{2008PhRvD..78l3506L} who derive the second-order Limber
expansion for general projections from 3D to 2D scalar fields in the spherical,
all-sky case. First, we use the identity of Bessel functions
%
\begin{equation}
  {\rm j}_\ell(x) = \sqrt{\frac{\pi}{2x}} {\rm J}_{\ell + 1/2}(x)
\end{equation}
%
in eq.~(\ref{eq:C_ell_full}), where ${\rm J}_\ell$ is the first-kind Bessel
function of order $\ell$. Next, \cite{2008PhRvD..78l3506L} solve integrals of
the form
%
\begin{equation}
  \int_0^\infty {\rm d} \chi f(\chi) {\rm J}_{\ell + 1/2}(k \chi)
  = \int_0^\infty {\rm d} x k^{-1} f(x/k) {\rm J}_{\ell + 1/2}(x)
  \label{eq:int_dchi}
\end{equation}
%
by performing a Taylor expansion of the function $f$ around $x = k \chi = \nu$, where
the Bessel function has its approximate maximum.

%%%%%%%%%%%%%%%%%%%%%%%%%%%%%%%%%%%%%%%%%%%%%%%%%%%%%%%%%%%%%%%
\subsubsection{Geometric mean cross-correlation power spectrum}
%%%%%%%%%%%%%%%%%%%%%%%%%%%%%%%%%%%%%%%%%%%%%%%%%%%%%%%%%%%%%%%

To separate the $k$- and $\chi, \chi^\prime$-terms in (\ref{eq:C_ell_full}), we
first approximate the matter power cross-spectrum between two distances as
geometric mean of the two involved distances \cite{2005PhRvD..72b3516C,2016arXiv161200770K},
%
\begin{equation}
 P_{\rm m}(k; \chi, \chi^\prime) = \sqrt{ P_{\rm m}(k; \chi) P_{\rm m}(k; \chi^\prime) }
  \label{eq:P_geom_mean}
\end{equation}
%
With this, eq.~(\ref{eq:C_ell_full}) is written as
%
\begin{align}
  C^\gamma_{ij}(\ell) \approx & \, \ellbar^2(\ell, 2) \, \pref^2
                \int_0^\infty \frac{{\rm d} k}{k^3} \,
                % P_{\rm m}(k) \,
                \int_{0}^\infty \frac{{\rm d} \chi}{\chi^{3/2}} \sqrt{P_{\rm m}(k; \chi)}
                \frac{q_i(\chi)}{a(\chi)} {\rm J}_{\ell+1/2}(k \chi)
                \nonumber \\
                 & \times
                \int_{0}^\infty \frac{{\rm d} \chi^\prime}{{\chi^\prime}^{3/2}}
                \sqrt{P_{\rm m}(k; \chi^\prime)} \frac{q_j(\chi^\prime)}{a(\chi^\prime)} {\rm J}_{\ell+1/2}(k \chi^\prime)
  \label{eq:C_ell_full_Pk}
\end{align}
%
We replaced the upper limits of the distance integral with infinity. \mk{Comment on this.}
Note that this equation has a preactor $\ellbar^2(\ell, 2)$ corresponding to a spin-2 field, in contrast
to \cite{2008PhRvD..78l3506L} who show calculations for a scalar field.

Following \cite{2008PhRvD..78l3506L} we expand to third order
%
\begin{equation}
  \lim_{\varepsilon \rightarrow 0} \int_0^\infty {\rm d} x \, {\rm e}^{-\epsilon (x - \nu)} g(x) {\rm J}_\nu(x)
  %\approx g(\nu) - \frac 1 2 g^{\prime\prime}(\nu) - \frac \nu 6 g^{\prime\prime\prime}(\nu) 
  \approx g(\nu) - \frac 1 2 \left.\frac{{\rm d}^2 g}{{\rm d} x^2}\right|_{x=\nu}
                 - \frac \nu 6 \left.\frac{{\rm d}^3 g}{{\rm d} x^3}\right|_{x=\nu} ,
\end{equation}
%
with $g(x) = k^{-1} f(k, \chi)$, $\chi=x/k$, and its derivatives $g^{(n)}(x) =
k^{-1-n} f^{(n)}(k, \chi)$ for a given $k$. In our case the projection kernel is
%
\begin{equation}
  f(k, \chi) = \sqrt{P_{\rm m}(k; \chi)} \, a^{-1}(\chi) \chi^{-3/2} q(\chi).
  \label{eq:f_LA08}
\end{equation}
%
(Add indices $i, j$ to $f$ and $q$ for the tomographic case.)
%
Replacing both distance integrals in (\ref{eq:C_ell_full_Pk}) by their Taylor-expansions around $\nu = k \chi$ and $\nu = k \chi^\prime$,
respectively, yields
%
\begin{align}
  C^\gamma_{ij}(\ell) \approx & \, \ellbar^2(\ell, 2) \, \pref^2
    \int_0^\infty \frac{{\rm d} k}{k^3} \, k^{-2}
    \left[ f_i(k, \chi) - \frac{1}{2 k^2} f_i^{\prime\prime}(k, \chi)
      - \frac{\nu}{6 k^3} f_i^{\prime\prime\prime}(k, \chi) + \ldots \right]
    %\nonumber \\
    %& \times
    \left[ f_j(k, \chi) - \frac{1}{2 k^2} f_j^{\prime\prime}(k, \chi)
    - \frac{\nu}{6 k^3} f_j^{\prime\prime\prime}(k, \chi) + \ldots \right]
  \label{eq:C_ell_limber2_dk}
\end{align}
%
Changing the integration to $\chi = \nu/k$ and collecting terms according to their $\nu$-dependence:
%
\begin{align}
  C^\gamma_{ij}(\ell) \approx & \, C^\gamma_{{\rm L1}, ij}(\ell) + C^\gamma_{{\rm L2}, ij}(\ell)
      \nonumber \\
    = & \frac{\ellbar^2(\ell, 2)}{\nu^4} \, \pref^2
    \int_0^\infty {\rm d} \chi \, \chi^3 \, % P_{\rm m}\left( k \right)
    %\Bigg\{
    \left\{
    (f_i f_j)(\nu/\chi, \chi)
    %\right.
      %\nonumber \\
    %&
    %\left.
     - \frac 1 {\nu^2} \left[ \frac{\chi^2}{2} \left( f^{}_i f^{\prime\prime}_j + f_i^{\prime\prime} f^{}_j \right)(\nu/\chi, \chi)
    + \frac{\chi^3}{6} \left( f^{}_i f^{\prime\prime\prime}_j + f^{\prime\prime\prime}_i f^{}_j \right)(\nu/\chi, \chi)
    \right]
    \right\}
    %\Bigg\}
  \label{eq:C_ell_limber12_dr}
\end{align}
%
The first term corresponds to the well-known first-order Limber approximation
\cite{1953ApJ...117..134L,1992ApJ...388..272K}, which is very widely used in
weak gravitational lensing. We retrieve the (spherical) standard expression by
inserting back the projection kernel (\ref{eq:f_LA08}) and the time-dependent
power spectrum,
%
\begin{align}
  C^\gamma_{{\rm L1}, ij}(\ell) = & \, \frac{\ellbar^2(\ell, 2)}{\nu^4} \, \pref^2 \int {\rm d} \chi \frac{ q_i(\chi) q_j(\chi) }{a^2(\chi)}
  P_{\rm m}\left(\frac{\nu}{\chi}; \chi\right).
  \label{eq:C_ell_limber1}
\end{align}
%
In the Limber approximation, modes between structures at different epochs do
not contribute to the single line-of-sight integration.

The second-order Limber term in (\ref{eq:C_ell_limber12_dr}) has an additional
$\nu^{-2}$-dependence, and is therefore strongly suppressed for large $\ell$.
%
\begin{align}
  C^\gamma_{{\rm L2}, ij}(\ell) = & - \frac 1 {\nu^2} \, \frac{\ellbar^2(\ell, 2)}{\nu^4} \, \pref^2
    \int {\rm d} \chi \, \chi^{2} a^{-2} \sqrt{P_{\rm m}\left(\frac\nu\chi; \chi\right)}
    %\nonumber \\
    %& \times
    \frac{a(\chi)}{2} \chi^{3/2} \left[ q^{}_i f^{\prime\prime}_j + f^{\prime\prime}_i q^{}_j
      + \frac{\chi}{3} \left( q^{}_i f^{\prime\prime\prime}_j + f^{\prime\prime\prime}_i q^{}_j
      \right)
    \right](\nu/\chi, \chi)
  \label{eq:C_ell_limber2_dr} 
\end{align}
%
The higher-order derivatives of the filter functions have to be computed
numerically in the general case. These suffer from numerical noise and are
sensitive to the set-up, for example the step size. The tabulation and
interpolation of those derivatives is time-taking since they depend on two
arguments, $\nu$ and $\chi$. In the following section, we will separate the
$k$- and $\chi$-dependend parts of the power spectrum to make the numerical
derivatives faster and more smooth.

%%%%%%%%%%%%%%%%%%%%%%%%%%%%%%%%%%%%%%%%%%%%%%%%%%%%%%%%%%%%%%%
\subsubsection{Time-independent power spectrum separation}
%%%%%%%%%%%%%%%%%%%%%%%%%%%%%%%%%%%%%%%%%%%%%%%%%%%%%%%%%%%%%%%

To further develop (\ref{eq:P_geom_mean}), we divide out the growth factor to
define a redshift-independent power spectrum $P_{\rm m}(k)$,
%
\begin{equation}
 P_{\rm m}(k, \chi) =: D_+(\chi) P_{\rm m}(k).
  \label{eq:P_k_chi_sep_s}
\end{equation}
%
This is not exact, since some non-linear prescriptions of the power spectrum
depend on redshift other than the growth factor, for example when computing the
non-linear scale $k_{\rm NL}(z)$ for \texttt{halofit}.

With this approximation, the $k$-independent term factors out from
(\ref{eq:f_LA08}), and we define the separated kernel function $f_{\rm s}$ as
%
\begin{equation}
  f_{\rm s}(\chi) = D_+(\chi) a^{-1}(\chi) \chi^{-3/2} q(\chi).
  \label{eq:f_LA08_s}
\end{equation}
%
and does not need to be considered when taking the derivatives with respect to comoving distances.
The second-order Limber term of the shear power spectrum can then be simplified to
%
\begin{align}
  C^\gamma_{{\rm L2}, ij}(\ell) = & - \frac 1 {\nu^2} \, \frac{\ellbar^2(\ell, 2)}{\nu^4} \, \pref^2
    \int {\rm d} \chi \, \chi^{2} a^{-2} P_{\rm m}\left(\frac\nu\chi; \chi\right)
    %\nonumber \\
    %& \times
    \frac {a(\chi)}{2 D_+(\chi)} \chi^{3/2}
    \left[ q^{}_i f^{\prime\prime}_{{\rm s} j} + f^{\prime\prime}_{{\rm s}i} q^{}_j
      + \frac{\chi}{3} \left( q^{}_i f^{\prime\prime\prime}_{{\rm s}j} + f^{\prime\prime\prime}_{{\rm s}i} q^{}_j
      \right)
    \right](\chi)
  \label{eq:C_ell_limber2_dr_s}
\end{align}
%
We compute the numerical higher derivatives as follows. The functions $f_{\rm
s}(\chi)$ are fitted as power laws with index $\approx 1.5$. The fit is
performed between $\chi_{\rm min} = 0.001 {\rm Mpc}/h$ and $\chi_{\rm max} =
500 \mbox{Mpc}/h$. At larger comoving distances the kernel decreases faster
than a power law, so we exclude this range from the fit. Even though the
derivaties are larger due to the steeper decline, the associated errors are
very small since those scales are down-weighed by the kernel function $f_{\rm
s}$ itself. At $\chi = 500 \mbox{Mpc}/h$ the filter function is less than
$10^{-4}$ of its value of unity at Mpc$/h$.

%The error between filter and power-law fit
%becomes larger than $50\%$ only for scales larger than $1,500 \mbox{Mpc}/h$. On the angular
%scales where the second-order Limber approximation significantly contributes to the shear
%power spectrum, $\ell \le 100$, this corresponds to $k < 0.03 h/$Mpc, linear
%scales where the 3D power spectrum is small and does not strongly contribute to the lensing
%power spectrum.

\subsection{Flat sky}
\label{sec:Limber_flat_sky}

The extended flat-sky Limber approximation is readily derived from the
spherical case, by replacing the prefactor $\ellbar^2(\ell, 2)$ with $\ell^4$,
%
\begin{align}
  P^\gamma_{{\rm L1}, ij}(\ell) = & \, \frac{\ell^4}{\nu^4} \, \pref^2 \int {\rm d} \chi \frac{ q_i(\chi) q_j(\chi) }{a^2(\chi)}
  P_{\rm m}\left(k = \frac{\nu}{\chi}; \chi\right);
  \label{eq:P_ell_limber1}
  \\
    P^\gamma_{{\rm L2}, ij}(\ell) = & - \frac 1 {\nu^2} \, \frac{\ell^4}{\nu^4} \, \pref^2
    \int {\rm d} \chi \, \chi^{2} a^{-2} P_{\rm m}\left(k = \frac\nu\chi; \chi\right)
    %\nonumber \\
    %& \times
    \frac{a(\chi)}{2 D^{-1}_+(\chi)} \chi^{3/2} \left[ q^{}_i f^{\prime\prime}_{{\rm s}j} + f^{\prime\prime}_{{\rm s}i} q^{}_j  
      + \frac{\chi}{3} \left( q^{}_i f^{\prime\prime\prime}_{{\rm s}j} + f^{\prime\prime\prime}_{{\rm s}i} q^{}_j
      \right)
    \right](\chi)
  \label{eq:P_ell_limber2}
\end{align}
%
Different further approximations are commonly made for the prefactor $p(\ell) = \ell^4/\nu^4$ and $\nu(\ell)$:
%
\begin{enumerate}
  \item $p(\ell) = 1$, this corresponds to $\nu(\ell) = \ell$, which is the \emph{standard} Limber approximation.
    Until recently, i.e.~for all pre-2014 CFHTlenS results, this was the approximation of choice.
  \item $p(\ell) = \ell^4/(\ell + 1/2)^4$. This corresponds to the \emph{extended} Limber
    approximation with $\nu(\ell) = \ell + 1/2$; however the following case is typically employed:
  \item $p(\ell) = 1$, but as the argument of the 3D power spectrum $\nu = \ell + 1/2$. This is
    a \emph{hybrid} between
    standard and extended Limber approximation, and was used in \cite{KiDS-450,joudaki/etal:2016}. As is shown
    later, this is a much better approximation to the full projection than case (ii).
\end{enumerate}

% (i)  Standard Flat
% (ii) Extended Flat
% (iii) Extended Flat Hybrid

%%%%%%%%%%%%%%%%%%%%%%%%%%%%%%%%%%%%%%%%%%%%%%%%%%%%%%
\section{Results}
\label{sec:results}
%%%%%%%%%%%%%%%%%%%%%%%%%%%%%%%%%%%%%%%%%%%%%%%%%%%%%%

%%%%%%%%%%%%%%%%%%%%%%%%%%%%%%%%%%%%%%%%%%%%%%%%%%%%%%
\subsection{Comparison of the approximations for the lensing power spectrum}
\label{sec:comp}
%%%%%%%%%%%%%%%%%%%%%%%%%%%%%%%%%%%%%%%%%%%%%%%%%%%%%%

In Fig.~\ref{fig:Cl_cases} we plot the full spherical projection of the shear
power spectrum in comparison to shear power spectra derived assuming the range
of different approximations listed in Table \ref{tab:cases}.  The adopted
redshift distribution corresponds to CFHTLenS \cite{CFHTLenS-2pt-notomo} and we
assume their best-fit flat $\Lambda$CDM cosmology with $\Omega=0.279$,
$\Omega_b=0.046$, $\sigma_8=0.70$, $H_0=0.701$, $n_s=0.96$. For $\ell > 100$ we
find that all shear power spectra predictions agree with the full spherical
solution to better than one percent, with the majority of the approximations
tested accurate to better than 0.1 per cent.   

Considering first the flat-sky cases, the standard first-order Limber
approximation (L1Fl), that was adopted for all pre-2014 CFHTLenS analyses, we
find to be accurate to better than 10\% for $\ell>3$, converging slowly to the
true projection with percent level precison at $\ell>100$. For the extended
Limber approximations `hybrid' cases (ExtL1FlHyb and ExtL2FlHyb), despite
decreased accuracy for $\ell < 8$ in comparison to the standard first-order
Limber case, the errors with respect to the true power spectrum decrease much
faster, as $\ell^{-2}$, such that percent-level precision is reached at
$\ell>15$.  The first-order extended Limber approximation `hybrid' case
(ExtL1FlHyb) was adopted by both \citet{joudaki/etal:2016} and
\cite{KiDS-450}\footnote{We confirm that there is a typographical error in
equation 4 of \cite{KiDS-450} which should include the extra term of `+0.5' in
$\nu(\ell)$ that was incorporated in the cosmological analysis.} 

The outlier in the flat-sky cases is the extended Limber approximation
(ExtL1Fl) which performs relatively poorly, and reach 10\% precision only at
$\ell > 100$. To our knowledge this form of the flat-sky approximation has not
been used in cosmic shear studies to date. The same slow convergence can be
observed for the corresponding second-order flat case, ExtL2Fl.

Including the spherical prefactor (the case ExtL1Sph) decreases the difference
by a factor of a few for $\ell < 5$. Going to second-order (ExtL2Sph) yields
percent accuracy down to $\ell = 4$.

\mk{Not sure whether this following paragraph should be left in the draft: One
notes that the difference between the full projection and second-order Limber
does not scale as expected with $\ell^{-4}$, but rather with $\ell^{-2}$. This
might be due to numerical noise that creates residual terms at the ${\cal
O}(\ell^{-2})$ which do not cancel. Also note that the numerical noise for
$\ell > 20$ is due to numerical integration errors for the full projection.}


\begin{figure}

  \begin{center}
    \resizebox{1.0\hsize}{!}{
      \includegraphics{figures/P_kappa_limber_comp}
      \includegraphics{figures/P_kappa_limber_delta}
    }
  \end{center}

  \caption{Shear power spectrum for different approximation. Limber to first order: standard with flat-sky (L1Fl),
        extended for flat sky (ExtL1Fl), extended hybrid for flat sky (ExtL1FlHyb),
        and extended in the spherical expansion (ExtL1Sph);
        second-order Limber approximations: extended flat sky (ExtL2Fl), extended hybrid flat sky (ExtL2FlHyb),
        and extended spherical expansion (ExtL2Sph); full (exact) spherical projection (FullSph).
        The left panels show the total power spectrum.
        See Table \ref{tab:cases} for more details on the different approximations.     
        }

  \label{fig:Cl_cases}

\end{figure}


% nicaea            Kil+17      'Comment paper' Plot color    Kit+16                Plot color    my comment
% limber            L1Fl        Kst             black         Limber+Tl=1+Flat-sky  blue
% limber_la08       ExtL1Fl     ELF             blue
% limber_la08_hyb   ExtL1FlHyb  ESt             magenta                                           prefactor~1 despite nu=ell+1/2
% limber_la08_sph   ExtL1Sph    ELS             cyan
% limber2_la08      ExtL2Fl
% limber2_la08_hyb  ExtL2FlHyb
% limber2_la08_sph  ExtL2Sph
% full              Full

\renewcommand{\baselinestretch}{1.3}
\begin{table}

  \label{tab:cases}

  \caption{Shear power spectrum approximations compared in this paper. 'ID' is the abbrevation used in the plots.
    The third column shows the power-spectrum prefactor $p(\ell)$.  \mk{The last column will be removed in the
        final version.}}

  \begin{tabular}{|l|l|c|c|c|c}
  \hline
  Case & ID & eq.~ & prefactor & $\nu(\ell)$ & Label [Comment paper] \\ \hline
  % 
  $1^{\rm st}$-order standard Limber, flat sky & L1Fl & (\ref{eq:P_ell_limber1}) + (i)
    & $1$ & $\ell$ & Kst \\ \hline
  %
  $1^{\rm st}$-order extended Limber, flat sky & ExtL1Fl & (\ref{eq:P_ell_limber1}) + (ii)
    & $\ell^4/(\ell+1/2)^4$ & $\ell + 1/2$ & ELF \\ \hline
  %
  $1^{\rm st}$-order extended Limber, hybrid, flat sky & ExtL1FlHyb & (\ref{eq:P_ell_limber1}) + (iii)
    & $1$ & $\ell + 1/2$ & ESt \\ \hline
  %
  $1^{\rm st}$-order extended Limber, spherical & ExtL1Sph & (\ref{eq:C_ell_limber1})
    & $\ellbar^2(\ell, 2)/(\ell+1/2)^4$ & $\ell+1/2$ & ELS \\ \hline
  %
  $2^{\rm st}$-order extended Limber, flat sky & ExtL2Fl &  (\ref{eq:P_ell_limber1}) + (\ref{eq:P_ell_limber2}) + (iii)
    & $\ell^4/(\ell+1/2)^4$ 
    & $\ell+1/2$ & - \\ \hline
  %
  $2^{\rm st}$-order extended Limber, hybrid, flat sky & ExtL2FlHyb &  (\ref{eq:P_ell_limber1}) + (\ref{eq:P_ell_limber2}) + (iii)
    & $1$ 
    & $\ell+1/2$ & - \\ \hline
  %
  $2^{\rm st}$-order extended Limber, spherical & ExtL2Sph & (\ref{eq:C_ell_limber1})+(\ref{eq:C_ell_limber2_dr_s})
    & $\ellbar^2(\ell, 2)/(\ell+1/2)^4$ & $\ell+1/2$ & - \\ \hline
  %
  Full spherical & FullSph & (\ref{eq:C_ell_full}) &
      $\ellbar^2(\ell, 2)$ & - & - \\ \hline
  %
  \end{tabular}

\end{table}
\renewcommand{\baselinestretch}{1}

%%%%%%%%%%%%%%%%%%%%%%%%%%%%%%%%%%%%%%%%%%%%%%%%%%%%%%
\subsection{Effects on the shear correlation function}
\label{sec:comp_xi}
%%%%%%%%%%%%%%%%%%%%%%%%%%%%%%%%%%%%%%%%%%%%%%%%%%%%%%



Most cosmic shear analysis to date have been performed on real-space
correlation statistics, since these can be measured direcaly from an observed
galaxy shape catalogue. The basic quantity is the two-point correlation
function \citep{miraldaescude:1991}, which is given by
%
\begin{eqnarray}
  \xi_+(\theta) 
  % &
  = \frac 1 {2\pi} \int {\rm d} \ell \, \ell {\rm J}_0(\ell
   \theta)
  P^\gamma(\ell);
  \quad
  %\nonumber \\
   %
   \xi_-(\theta)
  %&
  = \frac 1 {2\pi} \int
   {\rm d} \ell \, \ell {\rm J}_4(\ell \theta)
  P^\gamma(\ell).
  %
   \label{eqn:xiGG}
\end{eqnarray}
%
where ${\rm J}_\nu$ first-kind Bessel function of order $\nu$. The flat-sky
shear power spectrum $P^\gamma$ an be related to the underlying matter power
spectrum through equation \ref{eq:P_ell_limber1} when adopting a first-order
extended Limber approximation, or equation~\ref{eq:P_ell_limber2} when adopting
a second-order extended Limber approximation \citep[for more details see][and
references therein]{miraldaescude:1991, kaiser:1992,
bartelmann/schneider:2001}.

On a sphere, correlation functions cannot be related to the power spectrum by a
Hankel transform in equation~\ref{eqn:xiGG}, as the spherical power spectrum is
formally not defined for non-integer $\ell$ \citep[see][for alternative
spherical-sky formulae for the two-point correlation
function]{2005PhRvD..72b3516C}. In particular, the spherical Bessel functions
necessary to compute the full projection (\ref{eq:C_ell_full}) is defined only
for integer order $\ell$. \mk{Is this really true?}

However, in the Limber approximations the Bessel functions expanded to by
polynomials, allowing for arbitrary wave modes $\ell$. The spherical prefactor
$\ellbar(\ell, 2)$ (\ref{eq:ellbar}) can be generalised to non-integer
arguments and is positive for $\ell \le 2$. We can thus use the spherical power
spectrum in the Limber approximations for the Hankel transformation to compute
the two-point correlation functions, This is not entirely consistent, but
serves the purpose of comparison to the flat-sky case.

In addition, as we have shown in section~\ref{sec:comp} that the spherical
second-order extended Limber approximation provides a percent-level precision
representation of the full spherical projection for $\ell > 3$, we do not
present predictions for the two-point shear correlation function using the full
spherical projection, instead choosing the spherical second-order extended
Limber approximation (Ext2Sph) as our \mk{reference} in this case.

Figure.~\ref{fig:xi_pm_K16} presents the two-point correlation functions
$\xi_+$ (left) and $\xi_-$ (right) using the different cases for the shear
power spectrum listed in Table~\ref{tab:cases}. The adopted CFHTLenS redshift
distribution and fiducial cosmological model are the same as in
Figure.~\ref{fig:Cl_cases}. As is clear by the red dotted curve, using the
Planck cosmology \citep{2015arXiv150201589P} induces a much larger change in
the amplitude of the shear correlation function than the different projection
methods.


\begin{figure}

  \begin{center}
    \resizebox{1.0\hsize}{!}{
      \includegraphics{figures/xi_p_comp}
      \includegraphics{figures/xi_m_comp}
    }

    \resizebox{1.0\hsize}{!}{
      \includegraphics{figures/xi_p_delta}
      \includegraphics{figures/xi_m_delta}
    }
  \end{center}

  \caption{Shear correlation functions $\xi_+$ (\emph{left}) and $\xi_-$ (\emph{right}).
    The lower panels shows the relative differences to the case ExtL2Sph,  see Table \ref{tab:cases} for the different
    approximations and their symbols.
    The theoretical model corresponds to the CFHTLenS best-fit parameters with
    $\Omega_{\rm m} = 0.279$, $h=0.701$, and $\sigma_8 = 0.79$ \citep{2015arXiv150201589P},
    except the red dashed curve with adapts a Planck-best fit
    cosmology with $\Omega_{\rm m} = 0.3$, $h=0.67$ and $\sigma_8 = 0.83$ \citep{2015arXiv150201589P}.
  }

  \label{fig:xi_pm_K16}

\end{figure}


%%%%%%%%%%%%%%%%%%%%%%%%%%%%%%%%%%%%%%%%%%%%%%%%%%%%%%
\subsection{Application to CFHTLenS data}
\label{sec:cfhtlens}
%%%%%%%%%%%%%%%%%%%%%%%%%%%%%%%%%%%%%%%%%%%%%%%%%%%%%%


%\subsubsection{CFHTLenS updates and advances since 2014}

The Canada-France-Hawaii Telescope Lensing Survey (CFHTLenS) represented a
major step forward for the field of weak gravitational lensing, in terms of
improved accuracy in data reduction \citep{CFHTLenS-data}, the implementation
of PSF-Gaussianised matched multi-band photometry
\citep{CFHTLenS-photoz}, cross-correlation clustering analysis between
photometric redshift slices to verify tomographic redshift distributions
\citep{CFHTLenS-2pt-tomo}, accurate calibrated shape measurements
\citep{CFHTLenS-shapes} and a full suite of informative systematic tests to
select a clean data set \citep{CFHTLenS-sys}. Since the public release
of this survey in 2013, the community has continued to scrutinise and advance
our understanding of CFHTLenS by identifying a number of areas where analyses
could improve:
%
\begin{itemize}
%
 \item{\citet{2016MNRAS.463.3737C} identified significant biases in the tomographic
photometric redshift distributions using a more effective clustering analysis,
in comparison to \citet{CFHTLenS-2pt-tomo}, by incorporating newly overlapping
spectroscopic data from the Sloan Digital Sky Survey.  The conclusion of this
work was that any re-analysis of CFHTLenS should include systematic error terms
to account for bias and scatter, with a prediction that accounting for these
biases would {\it reduce} the recovered amplitude of $\sigma_8$ by $\sim
4$\%. Additional new techniques to calibrate the redshift distribution of tomographic
bins was introduced recently in \cite{KiDS-450}.}
%
\item{The CFHTLenS tomographic cosmological analysis was then revisited by
\citet{joudaki/etal:2016} in order to include a full redshift error analysis
based on the results from \citet{2016MNRAS.463.3737C}.  The impact of
correcting for these biases, including their associated errors, served to
reduce the overall constraining power of the survey and hence also the tension
between CFHTLenS and CMB constraints.}
%
 \item{\cite{asgari/etal:2017} used the stringent COSEBI statistic
\citep{COSEBIs} to identify significant non-lensing B-mode distortions when the
CFHTLenS data was split into tomographic slices.}
%
\item{\citet{2015MNRAS.454.3500K} showed that the CFHTLenS shear calibration
corrections derived in \citet{CFHTLenS-shapes} were underestimated as a result
of an imperfect match between the galaxy populations in the data and image
simulations.}
%
\item{\citet{2016arXiv160605337F} demonstrated that the CFHTLenS data would
have been subject to a weight bias that favours galaxies that are more
intrinsically oriented with the point-spread function.  They also showed that
the impact of calibration selection biases, that were not considered in
\citet{CFHTLenS-shapes}, would have lead to the over-correction of
multiplicative shear bias in the CFHTLenS analyses, by a few percent.}
%
\item{\citet{joudaki/etal:2016} updated the CFHTLenS covariance matrices using
larger-box numerical simulations that were less subject to the lack of power on
large scales. A complementary accurate estimate of the covariance matrix using
analytical methods will be published soon (Joachimi et al.~in prep.}
%
\item{\cite{2012ApJ...761..152T} provided a more accurate non-linear power
spectrum correction than that used in the original CFHTLenS analyses, and the
halo model from \cite{2015MNRAS.454.1958M} allowed for simultaneous modelling
of baryonic modifications to the non-linear power spectrum.} 
%
\end{itemize}
%
All these advances in our understanding were incorporated and accounted for in
the recent KiDS cosmic shear analysis \citep{KiDS-450} which reports a $2.3
\sigma$ tension with Planck.  Efforts are now underway to fully re-analyse
CFHTLenS using the advanced KiDS analysis pipeline with revised shape
measurements and calibrations for the shear and photometric redshifts. Until
this analysis is complete we note that these known shortcomings with the
original CFHTLenS results impact in different ways the cosmological conclusions
that one can draw. As CFHTLenS has similar statistical power
to current weak lensing surveys, however, it nevertheless provides a very
useful testbed with which to demonstrate the impact of adopting different
approximations when constraining cosmological parameters.

%\subsubsection{Cosmological analysis setup}

In this work, we focus on the weak-lensing power spectrum projection, and
assess the impact of various approximations on cosmological constraints from
CFHTLenS. For consistency with the original analysis
\citep{CFHTLenS-2pt-notomo}, we adopt the same priors and non-linear power
spectrum corrections from \cite{2003MNRAS.341.1311S}.

We re-analyse the 2D CFHTLenS measurement of the two-point shear correlation
function $\xi_\pm(\theta)$ from \cite{CFHTLenS-2pt-notomo}, defined in
equation~(\ref{eqn:xiGG}). As in \cite{CFHTLenS-2pt-notomo} we fit both
components $\xi_+$ and $\xi_-$ between angular scales $\theta = 0.8$ and $350$
arc minutes, and use a $N$-body simulation estimate of the non-Gaussian
covariance including the cross-covariance between both components. Bayesian
Population Monte-Carlo parameter sampling is performed using the publicly
available software
\textsc{CosmoPMC}\footnote{\texttt{http://www.cosmostat.org/software/cosmopmc}}
\citep{WK09,KWR10}. The cosmological modelling part includes the various
lensing projections, calculated using the software library
\textsc{nicaea}\footnote{\texttt{http://www.cosmostat.org/software/nicaea}}.

%%%%%%%%%%%%%%%%%%%%%%%%%%%%%%%%%%%%%%%%%%%%%%%%%%%%%%
%\subsubsection{Cosmological parameter results}
%\label{ref:cosmo_results}
%%%%%%%%%%%%%%%%%%%%%%%%%%%%%%%%%%%%%%%%%%%%%%%%%%%%%%


For a first-order standard Limber flat-sky approximation (L1Fl) we find
$\sigma_8 (\Omega_{\rm m}/0.27)^{0.6} =0.787^{+0.031}_{-0.033}$, the same
result that was published in \cite{CFHTLenS-2pt-notomo}. Using the second-order
extended Limber flat-sky hybrid approximation (ExtL2FlHyb) results in $\sigma_8
(\Omega_{\rm m}/0.27)^{0.6} = 0.788 \pm 0.032$, a negligible change of the
amplitude that is well within the Monte-Carlo sampling noise. The largest
difference is measured with the depreciated case ExtL1Fl, for which the
recovered amplitude is larger by $16\%$ of the statistical error. \ch{These
negligible changes to the error bars were to be expected owing to the
high level of statistical noise and cosmic variance in comparison to the low-level impact of the various
approximations shown in Fig.~\ref{fig:Cl_cases}.}

Table \ref{tab:CFHTLenS_Sigma8} lists the mean and 68\% credible interval for
$\sigma_8 \Omega_{\rm m}^{0.6}$ for the various approximations to the
lensing power spectrum projections listed in Table~\ref{tab:cases} Note again
that these values do not represent the state-of-the-art cosmological results,
since many of the above listed analysis advancements made since 2013 have not
been taken into account. As an example of a significant effect, when using the
revised non-linear power spectrum of \cite{2012ApJ...761..152T} in place of
\cite{2003MNRAS.341.1311S}, there is a decrease of $0.6 \sigma$ with $\sigma_8
(\Omega_{\rm m}/0.27)^{0.6} =0.768^{+0.029}_{-0.031}$.

\ch{Considering the cosmological constraints from tomographic Kilo-Degree Survey (KiDS), we conclude that
these are robust to flat-sky and Limber approximations. The case ExtL1FlHyb that was used
for the analysis of KiDS data in \citet{KiDS-450} and \cite{joudaki/etal:2017} introduces
errors that are more than an order of magnitude lower than the cosmic variance
for that survey, and thus this approximation has a negligible impact on the
cosmological parameters.}



\renewcommand{\baselinestretch}{1.5}
\begin{table}
\begin{centering}
  
  \caption{\label{tab:CFHTLenS_Sigma8}Mean and 68\% confidence interval for 
  $\sigma_8 (\Omega_{\rm m}/0.27)^{0.6}$ and $\sigma_8 (\Omega_{\rm m}/0.3)^{0.6}$
  for various approximations to the lensing
  power spectrum projections listed in Table~\ref{tab:cases}.}

  \begin{tabular}{lcc} \hline
  ID         & $\sigma_8 (\Omega_{\rm m}/0.27)^{0.6}$ & $\sigma_8 (\Omega_{\rm m}/0.3)^{0.6}$ \\ \hline
  L1Fl       & $0.787^{+0.031}_{-0.033}$ & $0.739^{+0.029}_{-0.031}$ \\
  ExtL1Fl    & $0.792 \pm 0.032$ & $0.744 \pm 0.030$ \\
  ExtL1FlHyb & $0.788^{+0.031}_{-0.033}$ & $0.740^{+0.029}_{-0.031}$ \\
  ExtL2FlHyb & $0.788^{+0.031}_{-0.033}$ & $0.740^{+0.029}_{-0.031}$ \\
  ExtL2Sph   & $0.789^{+0.031}_{-0.032}$ & $0.740^{+0.029}_{-0.030}$ \\ \hline
  \end{tabular}

\end{centering}
\end{table}
\renewcommand{\baselinestretch}{1}




\subsection{Comparison of two-point shear statistics; the two-point correlation function, mass aperture statistic and COSEBIs}

\ch{Copied this over from original comment paper - I would like to include this material somewhere in the paper - this needs work though to make it fit.... TBD}

\citet{kilbinger/etal:2013} present a detailed comparison of cosmological constraints obtained from a range of different two-point shear statistics including the shear correlation function, $\xi_\pm$, the aperture-mass dispersion, $\langle M_ {\rm ap} \rangle ^2$ \citep{schneider/etal:1998}, and the COSEBIs, $E_n$ \citep{schneider/etal:2010}.  These statistics are linearly related to the power spectrum via integrals of the form,
%
\begin{align}
%\label{eqn:integ}
& E_n =\int_0^{\infty}\frac{\d \ell \ell}{2\pi} W_n(\ell) P(\ell)\;,\\ \nonumber
& \langle M_ {\rm ap} \rangle ^2(\theta)=\int_0^{\infty}\frac{\d \ell \ell}{2\pi} U^2_\theta(\ell) P(\ell)\;,
\end{align}
%
where $W_n(\ell)$ and $U^2_\theta(\ell)$ are defined in \cite{schneider/etal:2010} and \cite{schneider/etal:1998}. The corresponding equations for $\xi_\pm$ are given in equation~\ref{eqn:xiGG}.
Figure~\ref{fig:filters} shows the integrands of these statistics 
for two cases, normalised to their maximum value, where the integrands are of the form $\ell F(\ell) P(\ell)$. For $\xi_\pm$, $P(\ell)$ is equal to the sum of the E and B-mode power spectra, motivating the development of the aperture-mass dispersion statistic, $\langle M_ {\rm ap} \rangle ^2$.  This statistic is, however, a lossy conversion and is biased by small angular separations, where blending of galaxies makes shear measurement challenging \citep{kilbinger/etal:2006}.  The COSEBIs statistic tackles both these shortcomings.  The upper two panels of figure~\ref{fig:filters} show the integrands\footnote{Note that the lower left panel of Fig. 1 in \citet{kitching/etal:2016} shows an integrated form of this function for a maximum angular separation of $100'$. However, in \citet{kilbinger/etal:2013}, the data used in \citet{kitching/etal:2016}, $\theta$ is between $0.8'$ and $350'$.} of $\xi_\pm$ for $\theta=100'$ and $\theta=350'$. The lower middle panel in Figure~\ref{fig:filters} shows the COSEBIs integrands for two angular ranges, $[1',100']$ and $[0.8',350']$, where we only show the integrands for the lowest COSEBIs mode, $E_1$, as the higher modes generally probe larger $\ell$-modes.  Finally, the lowest panel shows the integrands of aperture mass dispersion statistics, for the same two maximum angular ranges. 

\begin{figure}%[!htp]
\begin{center}
\begin{tabular}{ccc}
\includegraphics[width=0.48\textwidth]{figures/IntegAll.pdf} \\
\end{tabular}
\caption{ \small{\label{fig:filters}. Integrand of $\xi_+$ (upper), $\xi_-$ (upper middle), $E_1$ (lower middle, E-COSEBIs) and $\langle M_{\rm ap} \rangle^2$ (lower panel).
All integrands are of the form $\ell F(\ell) P(\ell)$, where $F(\ell)$ is the corresponding weight-function
for each statistic and $P(\ell)$ is the E-mode convergence power spectrum, with the exception of $\xi_\pm$, for which
$P(\ell)$ is equal to the sum of the E and B-mode power spectra. 
Two cases are shown for each statistic as listed in each caption.
For the aperture mass statistic $\theta_{\rm max}=2\theta$ is shown. 
Note that higher order COSEBIs modes generally probe larger $\ell$-modes, 
hence here we only show the lowest mode $E_1$. All values are normalized with respect to their maximum value. }
}
\end{center}
\end{figure}

From Figure~\ref{fig:filters} we can see that the two-point cosmic shear statistics tested by \citet{kilbinger/etal:2013} exhibit different dependences between the angular scales sampled and the $\ell$-range probed.   
If a significant bias had been introduced at low-$\ell$ by using flat-sky and Limber approximations, we would then expect to see a systematic shift between the different two-point statistics with the COSEBIs statistic being essentially unaffected as it only includes modes with $\ell \gtrsim 20$.  This is found not to be the case with all three statistics finding $\sigma_8 (\Omega_m/0.27)^\alpha = 0.79$ with errors that range from $0.03$ to $0.06$ for the full five-parameter fit, and $\alpha$ ranging from $0.59$ to 0.7 \citep[see Table 5 of][]{kilbinger/etal:2013}.  This comparison further supports our argument that the approximations highlighted by \citet{kitching/etal:2016} have negligible impact for current surveys.



\section{Conclusions}

\section*{Acknowledgments}

The authors thank Tommaso Giannantonio, Sarah Bridle, Ami Choi, Donnacha Kirk, Lance Miller, Benjamin Joachimi, Chris Blake and Adam Amara for very helpful discussions.

\bibliographystyle{mnras}
\bibliography{astro}

%%%%%%%%%%%%%%%%%%%%%%%%%%%%%%%%%%%%%%%%%%%%%%%%%%%%%%
\begin{appendix}
%%%%%%%%%%%%%%%%%%%%%%%%%%%%%%%%%%%%%%%%%%%%%%%%%%%%%%

%%%%%%%%%%%%%%%%%%%%%%%%%%%%%%%%%%%%%%%%%%%%%%%%%%%%%%
\section{Derivation of the weak-lensing power spectra}
\label{sec:derivations_C}
%%%%%%%%%%%%%%%%%%%%%%%%%%%%%%%%%%%%%%%%%%%%%%%%%%%%%%

The following derivations are detailed e.g.~in \cite{2000PhRvD..62d3007H} and
\cite{2005PhRvD..72b3516C}.

\subsection{Spherical case}

\subsubsection{Lensing potential power spectrum}

To obtain the power spectrum of the lensing potential for a flat Universe, 
we insert the lensing projection (\ref{eq:psi}) for $K=0$ into the
inverse harmonics expansion (\ref{eq:psi_harm_exp_inv}) and write the 3D potential
as its Fourier transform (\ref{eq:hatPhi_inv}) to get
%
\begin{equation}
  \psi_{\ell m} = \frac 2 {c^2} \int {\rm d} \Omega {\rm Y}^\ast_{\ell m}(\theta, \varphi)
    \int_0^{\chi_{\rm lim}} \frac{{\rm d}\chi}{\chi} q(\chi) \int \frac{{\rm d}^3 k}{(2\pi)^3} \hat \Phi(\vec k; \chi) {\rm e}^{-{\rm i} \vec k \vec r}.
\end{equation}
%
The 3D position vector $\vec r$ is a 3D position vector with polar coordinate
$r = \chi$ and polar angles $(\theta, \varphi)$. Similarly we denote with
$\theta_k, \varphi_k$ the polar angles of the 3D Fourier vector $\vec k$. We
insert the expansion of a plane wave into spherical harmonics,
%
% BBRV12 (41)
%
\begin{equation}
  {\rm e}^{{\rm i} \vec k \vec r} = 4 \pi \sum_{\ell=0}^{\infty} \sum_{m=-\ell}^{\ell}
    {\rm i}^\ell \, {\rm j}_\ell(k \chi)
    {\rm Y}_{\ell m}(\theta, \varphi) {\rm Y}_{\ell m}(\theta_k, \varphi_k) .
  \label{eq:wave_exp}
\end{equation}
%

Making use of the orthogonality of the spherical harmonics
%
% Rijt (6.15), Castro (B2).
%
\begin{equation}
  \int {\rm d} \Omega {\rm Y}^{}_{\ell m}(\theta, \varphi) {\rm Y}^\ast_{\ell^\prime m^\prime}(\theta, \varphi) = \delta_{\ell \ell^\prime} \delta_{m m^\prime},
  \label{eq:Yellm_ortho}
\end{equation}
%
the expression for $\psi_{\ell m}$ simplifies to
%
\begin{equation}
  \psi_{\ell m} = \frac {{\rm i}^\ell}{c^2 \pi^2} \int_0^{\chi_{\rm lim}} \frac{{\rm d}\chi}{\chi} q(\chi) \int {\rm d}^3 k \,
    \hat \Phi(\vec k; \chi) {\rm j}_\ell(k \chi) {\rm Y}_{\ell m}(\theta_k, \varphi_k).
  \label{eq:psi_ellm}
\end{equation}
%
We take the absolute square and use the definition of the 3D potential power spectrum (\ref{eq:p_phi}),
to arrive at expression (\ref{eq:C_ell_phi_Pphi}).

$P_{\rm m}$ is the 3D matter power cross-spectrum at line-of-sight comoving
Fourier mode $k$ and comoving distances $\chi$ and $\chi^\prime$. We denote
complex conjugation with superscript $^\ast$.
\mk{Peter's comment: What does the correlation between two epochs actually mean if they
are not on the backward light cone?}

The delta-function resolves one 3D Fourier
integral. We split the second integration into radial and spherical
coordinates, ${\rm d}^3 k = {\rm d} k k^2 {\rm d} \Omega_k$ and use once again
the orthogonality of the spherical harmonics to resolve the spherical integral. 
This leads to the potential power spectrum (\ref{eq:C_ell_phi_Pm}).




%%%%%%%%%%%%%%%%%%%%%%%%%%%%%%%%%%%%%%%%%%%%%%%%%%%%%%
\section{Discussion and comparison to previously published work}
\label{app:B}
%%%%%%%%%%%%%%%%%%%%%%%%%%%%%%%%%%%%%%%%%%%%%%%%%%%%%%

\subsection{Kitching et al.~2016}

\cite{2016arXiv161104954K} compute the full projection of the weak-lensing power
spectrum, which they present as spherical-radial representation the 3D shear field.
Our results (\ref{eq:C_ell_full}) correspond to their equations (7) and (8)
for a flat Universe and in the case of perfect
photometric redshifts, $p(z | z_{\rm p}) = \delta_{\rm D}(z - z_{\rm p})$, and
for a bin function that selects the redshift bin $i$, $W^{\rm SR}(z, z_{\rm p})
= 1_{z \in i-\mbox{th bin}}$. In \cite{2016arXiv161104954K} eq.~(7) is however the
factor $2/\pi$ missing.

In App.~A \cite{2016arXiv161104954K} derives the spherical and extended
Limber approximation starting from the full spherical projection. Note that
their filter function $q$ defined in (31) has an errornous factor of comoving
distance $r$, and an additional factor $\pi/2$ appears. \mk{This is probably worded
too strongly.}

We can not reproduce the amplitude of the differences between the cases,
neither for the power spectrum nor for the shear correlation function (see
Figs.~\ref{fig:Cl_cases} and \ref{fig:xi_pm_K16}).


\paragraph{Bernardeau, Bonvin, Van de Rijt, Vernizzi (2012); van de Rijt (2012)}

Eq.~(44) of \cite{2012PhRvD..86b3001B} is the non-tomographic $C(\ell)$ in
terms of the time-independent potential power spectrum $T^2(k) P(k)$, where $T$
is the transfer function; the growth factor $D$ is contained in their window
function $W$ (43) as growth-suppression factor $g(\chi) = D(\chi)/a(\chi)$
(defined after eq.~{24}). For a single source redshift, we find the identity
$W(\chi)$ = $q(\chi)\chi^{-1} D(\chi) a^{-1}(\chi)$.
\cite{2012PhRvD..86b3001B} is consistent with our
(\ref{eq:C_ell_full}), if we interpret the
scale factors at distances $\chi$ and $\chi^\prime$, respectively.

The PhD thesis of Van de Rijt \cite{vande2012} presents an explicit calculation
of the second-order Limber approximation for a fixed source redshift. They
carry out the derivatives of the kernels $f$ under the assumption of a constant
growth-suppression factor $D_+(a)/a$, and a power spectrum that separates $k$-
from $\chi$-terms. We show the consistency between their result (eq.~7.19) and
our (\ref{eq:C_ell_limber2_dr_s}), as follows.

We write (\ref{eq:C_ell_limber2_dr_s}) without tomography, and insert again the
time-independent power spectrum (\ref{eq:P_geom_mean}),
%
\begin{align}
  C^\gamma_{{\rm L2}}(\ell) = & - \frac 1 {\nu^2} \, \frac{\ellbar^2(\ell, 2)}{\nu^4} \, \pref^2
    \int {\rm d} \chi \, \chi^{2} P_{\rm m}\left(k = \frac\nu\chi\right)
    \nonumber \\
    & \times D_+(\chi) a^{-1}(\chi) \chi^{3/2} q(\chi) \left[ f^{\prime\prime} 
      + \frac{\chi}{3} f^{\prime\prime\prime} 
    \right](\chi)
  \label{eq:C_ell_limber2_dr_notomo}
\end{align}
%
For a fixed source comoving distance $\chi_{\rm S}$ the lensing efficiency is
$q(\chi) = (\chi_{\rm S} - \chi) \chi_{\rm S}^{-1}$. Assuming that
$D_+(\chi)/a(\chi) \approx g = \mbox{const}$, we get for the derivative terms,
with $f(\chi) = g \chi_{\rm S}^{-1} (\chi_{\rm S} - \chi) \chi^{-3/2}$,
%
\begin{equation}
  f^{\prime\prime}(\chi) + \frac{\chi}{3} f^{\prime\prime\prime}(\chi)
    = - \frac 1 8 \chi^{-5/2} \left( \frac 5 \chi
          + \frac 1 {\chi_{\rm S}} \right) \frac{D_+(\chi)}{a(\chi)}.
\end{equation}
%
Therefore,
%
\begin{align}
  C^\gamma_{{\rm L2}}(\ell) = \frac 1 {\nu^2} \, \frac{\ellbar^2(\ell, 2)}{\nu^4} \, \pref^2
    \int {\rm d} \chi \, \chi \, P_{\rm m}\left(k = \frac\nu\chi\right)
    \frac{q(\chi)}{8} \left( \frac 5 \chi + \frac 1 {\chi_{\rm S}} \right)
      \left(\frac{D_+(\chi)}{a(\chi)}\right)^2 
\end{align}
%
We now replace the matter with the potential power spectrum, ${\cal A}^2 P_{\rm m}(k) = k^{4} a^2 P_\Phi(k)$, and pull into the integral $\nu^{-4} = \chi^{-4} k^{-4}$. This yields
%
\begin{align}
  C^\gamma_{{\rm L2}}(\ell) = \frac 1 {\nu^2} \, \ellbar^2(\ell, 2)
    \int \frac{{\rm d} \chi}{\chi^2} \, P_\Phi\left(k = \frac\nu\chi\right)
    a^2(\chi) \frac{1}{8} \frac{\chi_{\rm S} - \chi}{\chi \chi_{\rm S}}
    \left( \frac 5 \chi + \frac 1 {\chi_{\rm S}} \right)
      \left(\frac{D_+(\chi)}{a(\chi)}\right)^2 .
\end{align}
%
Note the additional factor $a^2(\chi)$. This difference is accounted for by the
factor $a^{-1}$ in the propagation of primordial potential perturbations to
late times, see eq.~(6.22) in \cite{vande2012}, where the transition is done
using $g(a)$ instead of $D(a)$. \mk{Check this further.}


In Fig.~\ref{fig:L1L2E_Rijt} we reproduce Fig.~7.3 from \cite{vande2012} using
a similar set up (flat $\Lambda$ Universe with $\Omega_{\rm m} = 0.3, h=0.65,
\Omega_{\rm b} = 0.0461, \sigma_8 = 0.8, n_{\rm s} = 0.96$. All source galaxies
are at redshift $z_{\rm S} = 1$. The non-linear 3D matter power spectrum from
\cite{2012ApJ...761..152T} is used. The ratio of the first-and second- Limber approximated
power spectra to the full
projection shows excellent agreement at the sub-percent level.

\begin{figure}

  \begin{center}
    \resizebox{0.5\hsize}{!}{
      \includegraphics{figures/P_kappa_limber_delta_Rijt}
    }
  \end{center}

    \caption{Relative differences in percentage of spherical first- and second-order Limber
    shear power spectra with respect to the full projection
    as function of wave mode $\ell$, see Table \ref{tab:cases}. The redshift distribution
    and the cosmological parameters (see text) are chosen to match
    \citet{vande2012}, see their Fig.~7.3 for comparison.
    }

    \label{fig:L1L2E_Rijt}

\end{figure}



\paragraph{LoVerde \& Afshordi (2009)}

This paper introduces the extended Limber approximation to second order that we
apply in this work. Although they present the specific case of 2D galaxy
clustering, their calculations are general enough to apply to weak lensing.

Their eq.~(5) is a general spherical power spectrum of a scalar field,
projected from 3D to 2D via (4). Comparing their expressions with the
weak-lensing potential (\ref{eq:psi}), we can set $F_A(\chi) = F_B(\chi) = 2
c^{-2} D_+(\chi) q(\chi) \chi^{-1}$. With that, their eq.~(5) is identical to
our (\ref{eq:C_ell_phi_Pphi}).

The second-order Limber approximation in \cite{2008PhRvD..78l3506L} is presented in
eq.~(12). This is consistent with our first-order (\ref{eq:C_ell_limber1}) and
second-order (\ref{eq:C_ell_limber2_dr}) Limber approximation, when accounting for the difference between
lensing potential and shear 2D power spectrum, and 3D potential and matter power spectrum.


\paragraph{Schmidt (2009)}

In eq.~(9) \cite{2008PhRvD..78d3002S} write the lensing power spectrum in the
flat-sky limit. The Poisson factor $D_\phi$ is defined in their eq.~(5), see
also our eq.~\ref{eq:poisson}, and $D_{\rm m}$ is the growth factor. The filter
function $W_\kappa$ (eq.~10) can be written as
%
\begin{align}
  W_\kappa(z) = & \frac 1 {H(z)} \int_z^\infty {\rm d} z_{\rm s} W_{\rm L}(z_{\rm s}, z) W_{\rm g}(z)
              =  \frac 1 {H(z)} \chi \int_\chi^\infty {\rm d} \chi^\prime \frac{\chi^\prime - \chi}{\chi^\prime}
                n(\chi)
              \nonumber \\
              & \equiv \frac 1 {H(z)} \chi q(\chi),
\end{align}
%
using ${\rm d} z W_{\rm g}(z) = {\rm d} \chi n(\chi)$.
To write the redshift integrals (9) in \cite{2008PhRvD..78d3002S} over 
comoving distance, we make the variable transformation
%
\begin{equation}
  {\rm d} \chi = \frac{c {\rm d} z}{H(z)} .
\end{equation}
%
We recover our (\ref{eq:P_ell_gamma}).

\paragraph{Giannantonio et al. (2012)}

In eqs.~(25) and (26) \cite{2012MNRAS.422.2854G} write the flat-sky lensing
power spectrum. Their window function $W^{\varepsilon_i}$ defined in (25) for a
flat Universe equals $\pref \times q(\chi) / a(\chi)$, since ${\rm d} z {\rm d}
N(z)/{\rm d} z = {\rm d} \chi n(\chi)$; however there is a factor $r_K[r(z)]$
missing in the window function, which translates into a missing $r_K[r]
r_K[r^\prime]$ in the integral\footnote{Confirmed by T.~G., priv.~comm.}. This
also leads to an errorous $r_K^{-2}(r)$ in the Limber equation (27). With these
factors accounted for, we reproduce the expressions of \cite{2012MNRAS.422.2854G}.


\label{lastpage}

\end{appendix}

\end{document}

