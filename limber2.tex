\documentclass[useAMS,usenatbib]{mn2e} %
\pdfoutput=1

%% Packages
\usepackage{amsmath,amssymb,graphicx}
\usepackage{bm,color}% bold math
%\usepackage{ulem}
\usepackage{mathtools}
\usepackage{url}
\usepackage{footnote}
%\makesavenoteenv{tabular}
\makesavenoteenv{table}

\bibpunct{(}{)}{;}{a}{}{,}

\onecolumn

\pagestyle{empty}
\usepackage{psfrag, epsfig}


\newcommand{\ellbar}{\hbox{\it \l}\,}
\newcommand{\pref}{{\cal A}}
\newcommand{\edth}{\,\eth\,}
\renewcommand{\vec}{\bm}

\newcommand{\mk}[1]{{\bf\textcolor{blue}{#1}}}



%%%%%%%%%%%%%%%%%%%%%%%%%%%%%%%%%%%%%%%%%%%%%%%%%%%%%%%%%%%%
\begin{document}
%%%%%%%%%%%%%%%%%%%%%%%%%%%%%%%%%%%%%%%%%%%%%%%%%%%%%%%%%%%%

\title[]
{Precision calculations of the cosmic shear power spectrum projection}

\author[M.~Kilbinger et~al.]
 {
  \parbox[h]{\textwidth}
  {
      Martin Kilbinger$^{1,2}$\thanks{E-mail: martin.kilbinger@cea.fr},
      Catherine Heymans,
      Peter Schneider,
      Shahab Joudaki,
      Marika Asgari,
      and other members of CFHTLenS and KiDS
      $\ldots$,
  }
  % 
  \vspace*{10pt} \\
  \hspace{-.1cm}$^1$ CEA/Irfu/SAp Saclay, Laboratoire AIM, 91191 Gif-sur-Yvette, France\\
  \hspace{-.1cm}$^2$ Institut d'Astrophysique de Paris, UMR7095 CNRS,
           Universit\'e Pierre \& Marie Curie, 98 bis boulevard Arago, 75014 Paris,
           France \\
 }

\voffset-0.50in


\date{\today}

\pagerange{\pageref{firstpage}--\pageref{lastpage}} \pubyear{2017}

\maketitle

\label{firstpage}


\begin{abstract}

We compute the spherical-sky weak-lensing power spectrum of the shear and
convergence. We discuss various approximations, such as flat-sky, first- and
second-order Limber equations for the projection. We show that the spherical
second-order Limber approximation agrees with the full projection at the
percent level for $\ell > 3$. for current surveys. The real-space shear
correlation functions show excellent agreement between all cases. The
consequence for parameter constraints are negligible for current surveys, which
we demonstrate using CFHTLenS data. In particular, the reported tension with
Planck CMB temperature anisotropy results cannot be alleviated. This is in
contrast to the recent claim made by \citet{2016arXiv161104954K}. In the spirit
of reproducible research, our numerical implementation of all approximations
and the full projection are publically available within the package
\texttt{nicaea} at \texttt{http://www.cosmostat.org/software/nicaea}.

\end{abstract}

\begin{keywords}
cosmological parameters -- methods: statistical
\end{keywords}

\input abbr-journals

%%%%%%%%%%%%%%%%%%%%%%%%%%%%%%%%%%%%%%%%%%%%%%%%%%%%%%%%%%%%
\section{Introduction}
\label{sec:intro}
%%%%%%%%%%%%%%%%%%%%%%%%%%%%%%%%%%%%%%%%%%%%%%%%%%%%%%%%%%%%

The measurement of weak gravitational lensing by large-scale structures
provides a powerful cosmological probe of dark matter, dark energy, and
modifications to gravity.  As such it is the primary science goal of several
current (KiDS, HSC, DES) and future (Euclid, LSST, WFIRST) large surveys.
Interest in the results from these surveys is high as statistically significant
deviations have been found between the cosmological parameter constraints from
the CMB Planck experiment \citep{2015arXiv150201589P} in comparison to weak
lensing constraints from both the Kilo-Degree Survey \citep[KiDS;][]{KiDS-450}
and the Canada-France-Hawaii Telescope Lensing Survey
\citep[CFHTLenS;][]{2016arXiv160105786J}.  If the source of this tension is not
a result of so-far unconsidered sources of systematic errors in one or all
experiments, extensions to the standard flat $\Lambda$CDM cosmological models
need to be considered. \citet{joudaki/etal:2017} have shown, for example, that
the tension can be resolved with an evolving dark energy model.

In the era of the upcoming large surveys that will provide measurements of
cosmic shear with unprecedented precision, one needs to revisit the theoretical
model predictions of the observables to make sure that the accuracy of the
models meet the upcoming data. In this paper, we examine the widely used Limber
approximation for the projected weak-lensing power spectrum. We consider
spherical coordinates and the flat-sky approximation, and compute the full
projection as well as first- and second-order Limber approximation. We show
that the latter is an excellent approximation of the full projection. Since it
involves only 1D integrals over the matter power specrum, it is very fast to
calculate numerically and can readily employed in Monte-Carlo sampling methods
to obtain precision constraints on cosmological parameters.



%%%%%%%%%%%%%%%%%%%%%%%%%%%%%%%%%%%%%%%%%%%%%%%%%%%%%%%%%%%%
\section{Weak-lensing projections and power spectra}
\label{sec:wl}
%%%%%%%%%%%%%%%%%%%%%%%%%%%%%%%%%%%%%%%%%%%%%%%%%%%%%%%%%%%%

In this section we review the basic weak-lensing projection expressions, and
compute lensing power spectra for a spherical case, and in the flat-sky
approximation. Derivations for those expressions can be found in the appendix.

%%%%%%%%%%%%%%%%%%%%%%%%%%%%%%%%%%%%%%%%%%%%%%%%%%%%%%%%%%%%
\subsection{The lensing potential}
\label{sec:psi}
%%%%%%%%%%%%%%%%%%%%%%%%%%%%%%%%%%%%%%%%%%%%%%%%%%%%%%%%%%%%


The lensing potential $\psi$ in the Born approximation is defined as the
projected 3D metric potential $\Phi$ along the line of sight
\citep{1998ApJ...498...26K,BS01},
%
% Hu00 (21,22) with minus sign
%
\begin{equation}
  \psi(\theta, \varphi) = \frac 2 {{\rm c}^2} \int_0^{\chi_{\rm lim}} \frac{{\rm d}\chi}{\chi}
    \Phi[\chi, f_K(\chi) \theta, f_K(\chi) \varphi; \chi] \, q(\chi),
  \label{eq:psi}
\end{equation}
%
where the lensing efficiency $q$ is given by
%
\begin{equation}
  q(\chi) = \int\limits_\chi^{\chi_{\rm lim}} {\rm d} \chi^\prime \, n(\chi^\prime)
    \frac{f_K(\chi^\prime - \chi)}{f_K(\chi^\prime)},
  %
  \label{eq:lens_efficiency}
\end{equation}
%
corresponding to a population of lensed galaxies with source redshift
distribution $n(z) {\rm d}z = n(\chi) {\rm d} \chi$. Here, ${\rm c}$ is the
speed of light, the projection is carried out over comoving distances $\chi$ up
to a limiting distance $\chi_{\rm lim}$, and $f_K$ is the the comoving angular
distance, which depend on the spatial curvature of the Universe $K$.
%
This assumes a homogeneous galaxy distribution without clustering, so that the
redshift distribution in this approximation does not depend on the position
$(\theta, \varphi)$. Accounting for this position dependence leads to
correction of weak-lensing quantities due to clustering of source galaxies with
other sources \citep{2002A&A...389..729S}, and with galaxies associated with
lens structures \citep{1998A&A...338..375B,H02}.

The 3D potential is related to the density contrast $\delta$ via the Poisson
equation. Assuming General Relativity, this relation is written in Fourier space as
%
\begin{align}
  \hat \Phi(\vec k; \chi) = & - \frac 3 2 \Omega_{\rm m} H_0^2 k^{-2} a^{-1}(\chi) \hat \delta(\vec k; \chi),
      \label{eq:poisson}
\end{align}
%
where $\Omega_{\rm m}$ is the matter density parameter, $H_0$ the Hubble constant, $\vec k$ a 3D Fourier wave
vector, and $a$ the scale factor with $a=1$ today.
The Fourier transform of the potential and its inverse are defined as
%
\begin{align}
  \hat \Phi(\vec k; \chi) = &  \int {\rm d}^3 r \, \Phi(\vec r; \chi) {\rm e}^{{\rm i} \vec k \vec r} .
  \label{eq:hatPhi}
  \\
  \Phi(\vec r; \chi) = &  \int \frac{{\rm d}^3 k}{(2\pi)^3}
      \hat \Phi(\vec k; \chi) {\rm e}^{-{\rm i} \vec r \vec k},
  \label{eq:hatPhi_inv}
\end{align}
%
where the integration range for both integrals is $\mathbb{R}^3$.


%%%%%%%%%%%%%%%%%%%%%%%%%%%%%%%%%%%%%%%%%%%%%%%%%%%%%%%%%%%%
\subsection{Lensing power spectra in the spherical case}
%%%%%%%%%%%%%%%%%%%%%%%%%%%%%%%%%%%%%%%%%%%%%%%%%%%%%%%%%%%%

%%%%%%%%%%%%%%%%%%%%%%%%%%%%%%%%%%%%%%%%%%%%%%%%%%%%%%%%%%%%
\subsubsection{Lensing potential power spectrum}
%%%%%%%%%%%%%%%%%%%%%%%%%%%%%%%%%%%%%%%%%%%%%%%%%%%%%%%%%%%%

Following \cite{2000PhRvD..62d3007H} we decompose the lensing potential
(\ref{eq:psi}) into a spherical harmonics expansion, in analogy of CMB
temperature, both of which are scalar functions on the sphere. This
decomposition and its inverse are
%
% Hu00 (23)
%
\begin{align}
  \psi(\theta, \varphi) = \sum_{\ell=0}^\infty \sum_{m=-\ell}^\ell \psi_{\ell m} {\rm Y}_{\ell m}(\theta, \varphi);
    \label{eq:psi_harm_exp}
    \\
    %
  \psi_{\ell m} = \int_{\mathbb{S}^2} {\rm d} \Omega \, \psi(\theta, \varphi) {\rm Y}^\ast_{\ell m}(\theta, \varphi).
  \label{eq:psi_harm_exp_inv}
\end{align}

To specify tomographic redshift bins $i=0\ldots N_z-1$, we introduce a family
of lensing efficiency functions $q_i$ defined by a corresponding family of
redshift distributions $n_i$ via eq.~(\ref{eq:lens_efficiency}). The resulting
lensing potential is denoted by $\psi_{\ell m, i}$. The tomographic power
spectrum of the lensing potential between two redshift bins $i$ and $j$,
$C_{ij}^\psi$ is then defined by
%
\begin{equation}
  \left\langle \psi^{}_{\ell m, i} \, \psi^\ast_{\ell^\prime m^\prime, j} \right\rangle
    = \delta_{\ell \ell^\prime} \delta_{m m^\prime} C^\psi_{ij}(\ell) .
  \label{eq:C_ell_psi}
\end{equation}
%
Using the properties of the spherical harmonics (see
App.~\ref{sec:derivations_C} for details) the power spectrum can be written as
%
\begin{align}
  C_{ij}^\psi(\ell) = & \frac 8 {c^4 \pi} 
  \int_0^{\chi_{\rm lim}} \frac{{\rm d}\chi}{\chi} q_i(\chi)
  \int_0^{\chi^\prime_{\rm lim}} \frac{{\rm d}\chi^\prime}{\chi^\prime} q_j(\chi^\prime)
  \int {\rm d} k k^2 \, {\rm j}_\ell(k \chi) {\rm j}_\ell(k \chi^\prime) P_\Phi(k; \chi, \chi^\prime)
  \label{eq:C_ell_phi_Pphi} \\
  = & \frac 8 \pi \pref^2
  \int_0^{\chi_{\rm lim}} \frac{{\rm d}\chi}{\chi} \frac{q_i(\chi)}{a(\chi)}
  \int_0^{\chi^\prime_{\rm lim}} \frac{{\rm d}\chi^\prime}{\chi^\prime} \frac{q_j(\chi^\prime)}{a(\chi^\prime)}
  \int \frac{{\rm d} k}{k^2} \, {\rm j}_\ell(k \chi) {\rm j}_\ell(k \chi^\prime) P_{\rm m}(k; \chi, \chi^\prime) ;
  \label{eq:C_ell_phi_Pm}
\end{align}
%
where for convenience we defined the normalisation constant $\pref$ as
%
\begin{equation}
  \pref = \frac 3 2 \Omega_{\rm m} \left(\frac{H_0} c\right)^2.
  \label{eq:pref}
\end{equation}

In the following section we will discuss the relations between of shear and
convergence to the lensing potential on the sphere, and derive the power
spectrum of the former two fields.

%%%%%%%%%%%%%%%%%%%%%%%%%%%%%%%%%%%%%%%%%%%%%%%%%%%%%%
\subsubsection{Shear power spectrum}
%%%%%%%%%%%%%%%%%%%%%%%%%%%%%%%%%%%%%%%%%%%%%%%%%%%%%%

The shear $\gamma = \gamma_1 + {\rm i} \gamma_2$ is related to the potential at
linear order by the trace-free part of the Jacobi matrix. The involved
differential operator on the sphere is called \emph{edth} derivative, $\edth$,
see \cite{2005PhRvD..72b3516C} for a in-depth mathematical discussion of this
concept. The edth operator $\edth$ ($\edth^\ast$) raises (lowers) the spin $s$
of an object. Twice applying this operator to the scalar (spin-0) potential
creates the spin-2 shear:
%
\begin{align}
  \gamma(\theta, \varphi) = & \frac 1 2 \edth \edth \psi(\theta, \varphi).
    \nonumber \\
  \gamma^\ast(\theta, \varphi) = & \frac 1 2 \edth^\ast \edth^\ast \psi(\theta, \varphi).
  \label{gamma_psi_spher}
\end{align}
%
To write the shear on the sphere in terms of the lensing potential $\psi$, we
insert the harmonics expansion of the potential (\ref{eq:psi_harm_exp}). This
requires to compute second derivatives of the spherical harmonic functions.
This opereration defines a new object, the \emph{spin-weighted spherical
harmonic} $_s{\rm Y}_{\ell m}$. The shear can be written on the sphere in terms
of these functions as a spherical harmonics multipole expansion with
coefficients $_{\pm 2} \gamma_{\ell m}$. This expansion together with its inverse
is
%
% Hu04 (A10), van de Rijt PhD (6.32)
%
\begin{align}
  (\gamma_1 \pm {\rm i} \gamma_2)(\theta, \varphi) = & \sum_{\ell m} \,\, _{\pm 2}\gamma_{\ell m} \; _{\pm 2}\!{\rm Y}_{\ell m}(\theta, \varphi);
  \label{eq:gamma_harm_exp}
    \\
  \, _2 \gamma_{\ell m} = & \int_{\mathbb{S}^2} {\rm d} \Omega \, \gamma(\theta, \varphi) \;  _2\!{\rm Y}^\ast_{\ell m}(\theta, \varphi);
    \nonumber \\
  \, _{-2} \gamma_{\ell m} = & \int_{\mathbb{S}^2} {\rm d} \Omega \, \gamma^\ast(\theta, \varphi) \,  _{-2}\!{\rm Y}^\ast_{\ell m}(\theta, \varphi).
  \label{eq:gamma_harm_exp_inv}
\end{align}
%
The spin-weighted spherical harmonics $_s\!{\rm Y}_{\ell m}$ that are the basis function
in the expansion of the shear (\ref{eq:gamma_harm_exp}) can be calculated via the relations
%
% Rijt (6.16); BBRV12 (9)
%
\begin{align}
  \ellbar(\ell, s) \; _s\!{\rm Y}_{\ell m}(\theta, \varphi) = & \edth^s {\rm Y}_{\ell m}(\theta, \varphi);
    \nonumber \\
  \ellbar(\ell, s) \; _{-s}\!{\rm Y}_{\ell m}(\theta, \varphi) = & (-1)^s \left(\edth^\ast\right)^s {\rm Y}_{\ell m}(\theta, \varphi),
  \label{eq:sYlm_def} 
\end{align}
%
with the spin prefactor \citep{2012PhRvD..86b3001B}
%
\begin{equation}
  \ellbar(\ell, s) = \sqrt{\frac{(\ell + s)!}{(\ell - s)!}}.
\end{equation} 
%
Inserting the lensing potential expansion (\ref{eq:psi_harm_exp}) into the
expression for the shear (\ref{gamma_psi_spher}), and using (\ref{eq:sYlm_def})
to compute the derivatives, we find for the shear expansion coefficients
\citep{2000PhRvD..62d3007H,2001astro.ph.11605T}
%
% Rijt after (7.5); Taylor01 (18); Hu00 (A10)
%
\begin{equation}
  _{\pm 2} \gamma_{\ell m} = \frac 1 2 \ellbar(\ell, 2) \psi_{\ell m}.
  \label{eq:gamma_ellm_phi_ellm}
\end{equation}
%
The two coefficients $_{+2} \gamma_{\ell m}$ and $_{-2} \gamma_{\ell m}$ are
identical since the potential $\phi$ is a real function.

The tomographic shear power spectrum, in analogy to (\ref{eq:C_ell_phi_Pm}), is defined by
%
\begin{equation}
  \left\langle _2\gamma^{}_{\ell m, i} \; {}_2\gamma^\ast_{\ell^\prime m^\prime, j} \right\rangle
    = \delta_{\ell \ell^\prime} \delta_{m m^\prime} C^\gamma_{ij}(\ell).
  \label{eq:C_ell_gamma}
\end{equation}
%
For a flat Universe, this is given by
%
\begin{align}
  C^\gamma_{ij}(\ell) = \frac 1 4 \ellbar^2(\ell, 2) \, C^\psi_{ij}(\ell)
                 = & \frac 2 \pi \, \ellbar^2(\ell, 2) \, \pref^2
                 \int_{0}^{\chi_{\rm lim}} \frac{\rm d \chi}{\chi} \frac{q_i(\chi)}{a(\chi)}
                \int_{0}^{\chi_{\rm lim}} \frac{\rm d \chi^\prime}{\chi^\prime}
                \frac{q_j(\chi^\prime)}{a(\chi^\prime)}
                %\nonumber \\
                %& \times
                \int_0^\infty \frac{{\rm d} k}{k^2} \, P_{\rm m}(k, \chi, \chi^\prime) \,
                {\rm j}_\ell(k \chi) \, {\rm j}_\ell(k \chi^\prime) .
  \label{eq:C_ell_full}
\end{align}
%

%%%%%%%%%%%%%%%%%%%%%%%%%%%%%%%%%%%%%%%%%%%%%%%%%%%%%%
\subsubsection{Convergence power spectrum}
%%%%%%%%%%%%%%%%%%%%%%%%%%%%%%%%%%%%%%%%%%%%%%%%%%%%%%

The convergence is related to the lensing potential on the sphere via the
product of spin-raising and spin-lowering edth operators, which are identical
to the spherical Laplacian differential operator.

%
\begin{equation}
  \kappa(\theta, \varphi) = \frac 1 2 \edth \edth^\ast \psi(\theta, \varphi) = \frac 1 2 \nabla^2 \psi(\theta, \varphi).
  \label{eq:kappa_psi_spher}
\end{equation}
%
The spherical harmonics are eigenfunctions of the Laplacian,
%
\begin{equation}
  \nabla^2 {\rm Y}_{\ell m}(\theta, \varphi) = - \ell (\ell + 1) {\rm Y}_{\ell m}(\theta, \varphi)
    = - \ellbar^2(\ell, 1) {\rm Y}_{\ell m}(\theta, \varphi),
  \label{eq:nabla_Ylm}
\end{equation}
%
The potential power spectrum is then similar to the shear power spectrum
(\ref{eq:C_ell_gamma}) with a different spherical prefactor,
%
\begin{equation}
  C^\kappa_{ij}(\ell) = \frac 1 4 \ellbar^4(\ell, 1) \, C^\psi_{ij}(\ell)
    = \frac{\ell (\ell+1)}{(\ell-1)(\ell+2)} C^\gamma_{ij}(\ell) .
  \label{eq:C_ell_kappa_full}
\end{equation}
%
The convergence power spectrum is thus larger than the shear power spectrum, by
10\% for $\ell=4$, 1\% for $\ell = 14$, and less than 0.1\% for $\ell>45$.


%%%%%%%%%%%%%%%%%%%%%%%%%%%%%%%%%%%%%%%%%%%%%%%%%%%%%%
\subsection{Flat-sky approximation}
%%%%%%%%%%%%%%%%%%%%%%%%%%%%%%%%%%%%%%%%%%%%%%%%%%%%%%

Most previous work has used lensing quantities approximated on a flat sky,
neglecting the sky curvature. This is a valid approach for past and current
survey areas that have extents up to 10 degrees or less. The correlation
functions from the observed shear have been calculated using spherical
coordinates, and the effect on large scales is not negligible \cite{FSHK08}.
Here we examine the effect of sky curvature on the theoretical models.

For a flat sky, the spherical harmonics expansions are approximated by Fourier
transforms. The flat-sky equivalent of eqs.~(\ref{eq:psi_harm_exp}) and
(\ref{eq:psi_harm_exp_inv}) are
%
\begin{align}
  \psi(\vec \vartheta) = & \int \frac{{\rm d}^2 \ell}{(2\pi)^2} \, {\rm e}^{-{\rm i} \vec \ell \vec \vartheta} \hat \psi(\vec \ell);
  \label{eq:psi_fourier}
  \\
  %
  \hat \psi(\vec \ell) = & \int {\rm d}^2 \vartheta \, {\rm e}^{{\rm i} \vec \ell \vec \vartheta} \psi(\vec \vartheta);
  \label{eq:psi_fourier_inv}
\end{align}
%
where $\vec \vartheta = (\theta, \varphi)$ is the vector describing a 2D angle on the sky.
Instead of a harmonics coefficients $\psi_{\ell m}$, the Fourier representation of the potential
$\hat \psi$ now depends on the vector $\vec \ell \in R^2$.

The power spectrum, the flat-sky version of
(\ref{eq:C_ell_psi}) is defined by
%
\begin{equation}
  \left\langle \hat \psi_i^{}(\vec \ell) \hat \psi_j^\ast(\vec \ell^\prime) \right\rangle
    = (2\pi)^2 \delta_{\rm D}(\vec \ell - \vec \ell^\prime) P^\psi_{ij}(\ell).
  \label{eq:P_ell_psi}
\end{equation}


\mk{I can derive a flat-sky shear power spectrum by replacing $\ellbar^2(\ell, 2)$ with $\ell^4$
in (\ref{eq:C_ell_full}), equivalent to replacing the edth operators with Euclidian derivatives.
However, this results in a different expression compared to other references, e.g.~\cite{2008PhRvD..78d3002S},
which is the following equation.}

The flat-sky power spectrum is
%
\begin{align}
  P_{ij}^\gamma(\ell) = & \frac 2 \pi \, \pref^2
                 \int_{0}^{\chi_{\rm lim}} {\rm d} \chi \chi \, \frac{q_i(\chi)}{a(\chi)}
                \int_{0}^{\chi_{\rm lim}} {\rm d} \chi^\prime\, \chi^\prime
                \frac{q_j(\chi^\prime)}{a(\chi^\prime)}
                \nonumber \\
                & \times \int_0^\infty {\rm d} k \, k^2 \, P_{\rm m}(k, \chi, \chi^\prime) \,
                {\rm j}_\ell(k \chi) \, {\rm j}_\ell(k \chi^\prime) .
  \label{eq:P_ell_gamma}
\end{align}
%

\mk{I can get this by placing the prefactor under the integral and
setting $\ell^4 = k^4 \chi^2 {\chi^\prime}^2$. This does not seem to be the proper way.
Other derivations of the shear power spectrum (e.g.\cite{BS01} also assume other approximations
such as small-angle and Limber, but I do not know how to derive this without Limber.}


\section{Second-order Limber approximation for weak lensing}
\label{sec:L2}

\subsection{Spherical case}

We follow \cite{2008PhRvD..78l3506L} who derive the second-order Limber
expansion for general projections from 3D to 2D scalar fields in the spherical,
all-sky case. First, we use the identity of Bessel functions
%
\begin{equation}
  {\rm j}_\ell(x) = \sqrt{\frac{\pi}{2x}} {\rm J}_{\ell + 1/2}(x)
\end{equation}
%
in eq.~(\ref{eq:C_ell_full}). Next, \cite{2008PhRvD..78l3506L} solve
integrals of the form
%
\begin{equation}
  \int_0^\infty {\rm d} \chi f(\chi) {\rm J}_{\ell + 1/2}(k \chi)
  = \int_0^\infty {\rm d} x k^{-1} f(x/k) {\rm J}_{\ell + 1/2}(x)
  \label{eq:int_dchi}
\end{equation}
%
by performing a Taylor expansion of the function $f$ around $x = k \chi = \nu$, where
the Bessel function has its approximate maximum. To separate the $k$- and
$\chi, \chi^\prime$-terms in (\ref{eq:C_ell_full}), we first approximate the
matter power cross-spectrum between two distances as geometric mean of the two
involved distances \cite{2016arXiv161200770K}, and divide out the growth factor
to define a redshift-independent power spectrum $P_{\rm m}(k)$,
%
\begin{equation}
 P_{\rm m}(k, \chi, \chi^\prime) = \sqrt{ P_{\rm m}(k, \chi) P_{\rm m}(k, \chi^\prime) },
    = D_+(\chi) D_+(\chi^\prime) P_{\rm m}(k).
  \label{eq:P_geom_mean}
\end{equation}
%
This is not exact, since some non-linear prescriptions of the power spectrum
depend on redshift other than the growth factor, for example when computing the
non-linear scale $k_{\rm NL}(z)$ for \texttt{halofit}.

With this, eq.~(\ref{eq:C_ell_full}) is written as
%
\begin{align}
  C^\gamma_{ij}(\ell) \approx & \, \ellbar^2(\ell, 2) \, \pref^2
                \int_0^\infty \frac{{\rm d} k}{k^3} \, P_{\rm m}(k) \,
                \int_{0}^\infty \frac{{\rm d} \chi}{\chi^{3/2}} \frac{D_+(\chi)}{a(\chi)} q_i(\chi) {\rm J}_{\ell+1/2}(k \chi)
                \nonumber \\
                 & \times
                \int_{0}^\infty \frac{{\rm d} \chi^\prime}{{\chi^\prime}^{3/2}}
                \frac{D_+(\chi^\prime)}{a(\chi^\prime)} q_j(\chi^\prime) {\rm J}_{\ell+1/2}(k \chi^\prime)
  \label{eq:C_ell_full_Pk}
\end{align}
%
We replaced the upper limits of the distance integral with infinity. \mk{Comment on this.}
Note that this equation has a preactor $\ellbar^2(\ell, 2)$ corresponding to a spin-2 field, in contrast
to \cite{2008PhRvD..78l3506L} who show calculations for a scalar field.

Following \cite{2008PhRvD..78l3506L} we expand to third order
%
\begin{equation}
  \lim_{\varepsilon \rightarrow 0} \int_0^\infty {\rm d} x \, {\rm e}^{-\epsilon (x - \nu)} g(x) {\rm J}_\nu(x) \approx g(\nu) - \frac 1 2 g^{\prime\prime}(\nu)
      - \frac \nu 6 g^{\prime\prime\prime}(\nu) 
\end{equation}
%
with $g(x) = k^{-1} f(\chi)$, $\chi=x/k$, and its derivatives $g^{(n)}(x) =
k^{-1-n} f^{(n)}(\chi)$. In our case the projection kernel is
%
\begin{equation}
  f(\chi) = D_+(\chi) a^{-1}(\chi) \chi^{-3/2} q(\chi).
  \label{eq:f_LA08}
\end{equation}
%
(Add indices $i, j$ to $f$ and $q$ for the tomographic case.)
%
Replacing both distance integrals in (\ref{eq:C_ell_full_Pk}) by their Taylor-expansions around $\nu = k \chi$ and $\nu = k \chi^\prime$,
respectively, yields
%
\begin{align}
  C^\gamma_{ij}(\ell) \approx & \, \ellbar^2(\ell, 2) \, \pref^2
    \int_0^\infty \frac{{\rm d} k}{k^3} \, P_{\rm m}(k) k^{-2}
    \left[ f_i(\chi) - \frac{1}{2 k^2} f_i^{\prime\prime}(\chi)
      - \frac{\nu}{6 k^3} f_i^{\prime\prime\prime}(\nu) + \ldots \right]
    %\nonumber \\
    %& \times
    \left[ f_j(\chi) - \frac{1}{2 k^2} f_j^{\prime\prime}(\chi)
    - \frac{\nu}{6 k^3} f_j^{\prime\prime\prime}(\nu) + \ldots \right]
  \label{eq:C_ell_limber2_dk}
\end{align}
%
Changing the integration to $\chi = \nu/k$ and collecting terms according to their $\nu$-dependence:
%
\begin{align}
  C^\gamma_{ij}(\ell) \approx & \, C^\gamma_{{\rm L1}, ij}(\ell) + C^\gamma_{{\rm L2}, ij}(\ell)
      \nonumber \\
    = & \frac{\ellbar^2(\ell, 2)}{\nu^4} \, \pref^2
    \int_0^\infty {\rm d} \chi \, \chi^3 \, P_{\rm m}\left( k \right)
    %\Bigg\{
    \left\{
    f_i(\chi) f_j(\chi)
    %\right.
      %\nonumber \\
    %&
    %\left.
     - \frac 1 {\nu^2} \left[ \frac{\chi^2}{2} \left( f^{}_i f^{\prime\prime}_j + f_i^{\prime\prime} f^{}_j \right)(\chi)
    + \frac{\chi^3}{6} \left( f^{}_i f^{\prime\prime\prime}_j + f^{\prime\prime\prime}_i f^{}_j \right)(\chi)
    \right]
    \right\}
    %\Bigg\}
  \label{eq:C_ell_limber12_dr}
\end{align}
%
The first term corresponds to the well-known first-order Limber approximation
\cite{1953ApJ...117..134L,1992ApJ...388..272K}, which is very widely used in
weak gravitational lensing. We retrieve the (spherical) standard expression by
inserting back the projection kernel (\ref{eq:f_LA08}) and the time-dependent
power spectrum,
%
\begin{align}
  C^\gamma_{{\rm L1}, ij}(\ell) = & \, \frac{\ellbar^2(\ell, 2)}{\nu^4} \, \pref^2 \int {\rm d} \chi \frac{ q_i(\chi) q_j(\chi) }{a^2(\chi)}
  P\left(k = \frac{\nu}{\chi}; \chi\right).
  \label{eq:C_ell_limber1}
\end{align}
%
In the Limber approximation, modes between structures at different epochs do
not contribute to the single line-of-sight integration.

The second-order Limber term in (\ref{eq:C_ell_limber12_dr}) has an additional
$\nu^{-2}$-dependence, and is therefore strongly suppressed for large $\ell$.
%
\begin{align}
  C^\gamma_{{\rm L2}, ij}(\ell) = & - \frac 1 {\nu^2} \, \frac{\ellbar^2(\ell, 2)}{\nu^4} \, \pref^2
    \int {\rm d} \chi \, \chi^{2} a^{-2} P_{\rm m}\left(k = \frac\nu\chi; \chi\right)
    %\nonumber \\
    %& \times
    \frac 1 2 D^{-1}_+(\chi) a(\chi) \chi^{3/2} \left[ q^{}_i f^{\prime\prime}_j + f^{\prime\prime}_i q^{}_j
      + \frac{\chi}{3} \left( q^{}_i f^{\prime\prime\prime}_j + f^{\prime\prime\prime}_i q^{}_j
      \right)
    \right](\chi)
  \label{eq:C_ell_limber2_dr} 
\end{align}
%
The derivatives of the filter functions have to be computed numerically in the general case.


\subsection{Flat sky}

The extended flat-sky Limber approximation is readily derived from the
spherical case, by replacing the prefactor $\ellbar^2(\ell, 2)$ with $\ell^4$,
%
\begin{align}
  P^\gamma_{{\rm L1}, ij}(\ell) = & \, \frac{\ell^4}{\nu^4} \, \pref^2 \int {\rm d} \chi \frac{ q_i(\chi) q_j(\chi) }{a^2(\chi)}
  P_{\rm m}\left(k = \frac{\nu}{\chi}; \chi\right);
  \label{eq:P_ell_limber1}
  \\
    P^\gamma_{{\rm L2}, ij}(\ell) = & - \frac 1 {\nu^2} \, \frac{\ell^4}{\nu^4} \, \pref^2
    \int {\rm d} \chi \, \chi^{2} a^{-2} P_{\rm m}\left(k = \frac\nu\chi; \chi\right)
    %\nonumber \\
    %& \times
    \frac 1 2 D^{-1}_+(\chi) a(\chi) \chi^{3/2} \left[ q^{}_i f^{\prime\prime}_j + f^{\prime\prime}_i q^{}_j  
      + \frac{\chi}{3} \left( q^{}_i f^{\prime\prime\prime}_j + f^{\prime\prime\prime}_i q^{}_j
      \right)
    \right](\chi)
  \label{eq:P_ell_limber2}
\end{align}
%
Different further approximations are commonly made for the prefactor $p(\ell) = \ell^4/\nu^4$ and $\nu(\ell)$.
%
\begin{enumerate}
  \item $p(\ell) = 1$, this corresponds to setting $\nu(\ell) = \ell$, and is the standard Limber approximation.
    Until recently, i.e.~for all pre-2014 CFHTlenS results, this was the approximation of choice.
  \item $p(\ell) = \ell^4/\nu^4(\ell)$ with $\nu(\ell) = \ell + 1/2$. This corresponds to the extended Limber
    approximation; however the following case is typically employed:
  \item $p(\ell) = 1$, but as the argument of the 3D power spectrum $\nu = \ell + 1/2$. This is a hybrid between
    standard and extended Limber approximation, and was used in \cite{KiDS-450,joudaki/etal:2016}. As is shown
    later, this is a much better approximation than case 2.
\end{enumerate}

%%%%%%%%%%%%%%%%%%%%%%%%%%%%%%%%%%%%%%%%%%%%%%%%%%%%%%
\section{Comparison of the approximations}
\label{sec:comp}
%%%%%%%%%%%%%%%%%%%%%%%%%%%%%%%%%%%%%%%%%%%%%%%%%%%%%%

In Fig.~\ref{fig:Cl_cases} we plot the full shear power spectrum and some of
the approximations. For $\ell > 100$ all power spectra coincide to better than one percent.
The extended first-order Limber approximation shows a better convergence to the true
power spectrum, the error decreases with $\ell^{-2}$. Including the spherical prefactor
decreases the difference by a factor of a few for $\ell < 5$.
Going to second-order yields percent accuracy down to $\ell = 4$.

\begin{figure}

  \begin{center}
    \resizebox{1.0\hsize}{!}{
      \includegraphics{P_kappa_limber_comp}
      \includegraphics{P_kappa_limber_delta}
    }
  \end{center}

  \caption{Shear power spectrum for different approximation. Limber to first order: for flat-sky (L1Fl),
        extended ($\ell + 1/2$) for flat sky (ExtL1Fl), and extended in the spherical expansion (ExtL1Sph);
        Second-order extended Limber spherical expansion (ExtL2Sph); full (exact) spherical projection (FullSph).
        The left (right) panels show the total (relative difference) power spectrum.
        See Table \ref{tab:cases} for more details on the different approximations. \mk{Right panel needs to be updated!
        ExtL1Fl curve is actually ExtL1FlHyb.}
      }

  \label{fig:Cl_cases}

\end{figure}


% nicaea            Kil+17      'Comment paper' Plot color    Kit+16                Plot color    my comment
% limber            L1Fl        Kst             black         Limber+Tl=1+Flat-sky  blue
% limber_la08       ExtL1Fl     ELF             blue
% limber_la08_hyb   ExtL1FlHyb  ESt             magenta                                           prefactor~1 despite nu=ell+1/2
% limber_la08_sph   ExtL1Sph    ELS             cyan
% limber2_la08      ExtL2Fl
% limber2_la08_hyb  ExtL2Fl
% limber2_la08_sph  ExtL2Sph
% full              Full

\renewcommand{\baselinestretch}{1.3}
\begin{table}[ht!]

  \label{tab:cases}

  \caption{Shear power spectrum approximations compared in this paper. \mk{The last column will be removed in the
        final version.}}

  \begin{tabular}{|l|l|c|c|c|c}
  \hline
  Case & ID & eq.~ & $p(\ell)$ & $\nu(\ell)$ & Label [Comment paper] \\ \hline
  % 
  $1^{\rm st}$-order standard Limber, flat sky & L1Fl & (\ref{eq:P_ell_limber1}) + (i)
    & $1$ & $\ell$ & Kst \\ \hline
  %
  $1^{\rm st}$-order extended Limber, flat sky & ExtL1Fl & (\ref{eq:P_ell_limber1}) + (ii)
    & $\ell^4/(\ell+1/2)^4$ & $\ell + 1/2$ & ELF \\ \hline
  %
  $1^{\rm st}$-order extended Limber, hybrid, flat sky & ExtL1FlHyb & (\ref{eq:P_ell_limber1}) + (iii)
    & $1$ & $\ell + 1/2$ & ESt \\ \hline
  %
  $1^{\rm st}$-order extended Limber, spherical & ExtL1Sph & (\ref{eq:C_ell_limber1})
    & $\ellbar^2(\ell, 2)/(\ell+1/2)^4$ & $\ell+1/2$ & ELS \\ \hline
  %
  $2^{\rm st}$-order extended Limber, flat sky & Ext2FlHyb & (\ref{eq:P_ell_limber2}) + (iii)
    & $\ellbar^2(\ell, 2)/(\ell+1/2)^4$ & $\ell+1/2$ & - \\ \hline
  %
  $2^{\rm st}$-order extended Limber, spherical & Ext2Sph & (\ref{eq:C_ell_limber1})+(\ref{eq:C_ell_limber2_dr})
    & $\ellbar^2(\ell, 2)/(\ell+1/2)^4$ & $\ell+1/2$ & - \\ \hline
  %
  Full spherical & FullSph & (\ref{eq:C_ell_full}) &
      $\ellbar^2(\ell, 2)$ & - & - \\ \hline
  %
  \end{tabular}

\end{table}
\renewcommand{\baselinestretch}{1}

In Fig.~\ref{fig:xi_pm_K16} we show the correlation functions using the different cases
for the shear power spectrum. \mk{Replace this figure, don't show CFHTLenS data?}

\begin{figure}

  \begin{center}
    \resizebox{1.0\hsize}{!}{
      \includegraphics{xi_p_comp}
      \includegraphics{xi_m_comp}
    }
  \end{center}

  %\caption{Shear correlation funtions $\xi_+$ (\emph{left panel}) and $\xi_-$ (\emph{right}) in the
    %%represenation as in \cite{2016arXiv161104954K}. See Table \ref{tab:cases} for the cases of different
    %approximations and their symbols. The  points with error bars are the data from
    %\cite{CFHTLenS-2pt-notomo}; the theotical predictions correspond to the best-fit parameters with
    %$\Omega_{\rm m} = 0.279$ and $\sigma_8 = 0.79$, except the red dashed curve with adapts a Planck-best fit
    %normalisation $\sigma_8 = 0.83$.}

  \label{fig:xi_pm_K16}

\end{figure}

%%%%%%%%%%%%%%%%%%%%%%%%%%%%%%%%%%%%%%%%%%%%%%%%%%%%%%
\section{Application to CFHTLenS data}
\label{sec:cfhtlens}
%%%%%%%%%%%%%%%%%%%%%%%%%%%%%%%%%%%%%%%%%%%%%%%%%%%%%%


\begin{table}

  \begin{tabular}{lc} \hline
  ID       & $\sigma_8 (\Omega_{\rm m}/0.3)^{0.6}$ \\ \hline
  L1Fl     & $0.769 \pm 0.031$ \\
  ExtL2Sph & $0.769 \pm 0.030$ \\ \hline
  \end{tabular}

\end{table}

\section*{Acknowledgments}

The authors thank Tommaso Giannantonio for very helpful discussions.

\bibliographystyle{mn2e}
\bibliography{astro}

%%%%%%%%%%%%%%%%%%%%%%%%%%%%%%%%%%%%%%%%%%%%%%%%%%%%%%
\begin{appendix}
%%%%%%%%%%%%%%%%%%%%%%%%%%%%%%%%%%%%%%%%%%%%%%%%%%%%%%

%%%%%%%%%%%%%%%%%%%%%%%%%%%%%%%%%%%%%%%%%%%%%%%%%%%%%%
\section{Derivation of the weak-lensing power spectra}
\label{sec:derivations_C}
%%%%%%%%%%%%%%%%%%%%%%%%%%%%%%%%%%%%%%%%%%%%%%%%%%%%%%

The following derivations are detailed e.g.~in \cite{2000PhRvD..62d3007H} and
\cite{2005PhRvD..72b3516C}.

\subsection{Spherical case}

\subsubsection{Lensing potential power spectrum}

To obtain the power spectrum of the lensing potential for a flat Universe, 
we insert the lensing projection (\ref{eq:psi}) for $K=0$ into the
inverse harmonics expansion (\ref{eq:psi_harm_exp_inv}) and write the 3D potential
as its Fourier transform (\ref{eq:hatPhi_inv}) to get
%
\begin{equation}
  \psi_{\ell m} = \frac 2 {c^2} \int {\rm d} \Omega {\rm Y}^\ast_{\ell m}(\theta, \varphi)
    \int_0^{\chi_{\rm lim}} \frac{{\rm d}\chi}{\chi} q(\chi) \int \frac{{\rm d}^3 k}{(2\pi)^3} \hat \Phi(\vec k; \chi) {\rm e}^{-{\rm i} \vec k \vec r}.
\end{equation}
%
The 3D position vector $\vec r$ is a 3D position vector with polar coordinate
$r = \chi$ and polar angles $(\theta, \varphi)$. Similarly we denote with
$\theta_k, \varphi_k$ the polar angles of the 3D Fourier vector $\vec k$. We
insert the expansion of a plane wave into spherical harmonics,
%
% BBRV12 (41)
%
\begin{equation}
  {\rm e}^{{\rm i} \vec k \vec r} = 4 \pi \sum_{\ell m} {\rm i}^\ell {\rm j}_\ell(k \chi)
    {\rm Y}_{\ell m}(\theta, \varphi) {\rm Y}_{\ell m}(\theta_k, \varphi_k),
  \label{eq:wave_exp}
\end{equation}
%
\mk{Comment on minus sign in exponential.}

Making use of the orthogonality of the spherical harmonics
%
% Rijt (6.15), Castro (B2).
%
\begin{equation}
  \int {\rm d} \Omega {\rm Y}^{}_{\ell m}(\theta, \varphi) {\rm Y}^\ast_{\ell^\prime m^\prime}(\theta, \varphi) = \delta_{\ell \ell^\prime} \delta_{m m^\prime},
  \label{eq:Yellm_ortho}
\end{equation}
%
the expression for $\psi_{\ell m}$ simplifies to
%
\begin{equation}
  \psi_{\ell m} = \frac {{\rm i}^\ell}{c^2 \pi^2} \int_0^{\chi_{\rm lim}} \frac{{\rm d}\chi}{\chi} q(\chi) \int {\rm d}^3 k \,
    \hat \Phi(\vec k; \chi) {\rm j}_\ell(k \chi) {\rm Y}_{\ell m}(\theta_k, \varphi_k).
  \label{eq:psi_ellm}
\end{equation}
%
We take the absolute square and use the definition of the 3D potential power spectrum,
%
\begin{align}
  \left\langle \Phi(\vec k; \chi) \Phi^\ast(\vec k^\prime; \chi^\prime) \right\rangle
    = & (2\pi)^3 \delta_{\rm D}(\vec k - \vec k^\prime) P_\Phi(k; \chi, \chi^\prime).
    %\nonumber \\
    %= & (2\pi)^3 \delta_{\rm D}(\vec k - \vec k^\prime) \pref^2 k^{-4} a^{-1}(\chi) a^{-1}(\chi^\prime)
      %P_{\rm m}(k; \chi, \chi^\prime).
  \label{eq:p_phi}
\end{align}
%
$P_{\rm m}$ is the 3D matter power cross-spectrum at line-of-sight comoving
Fourier mode $k$ and comoving distances $\chi$ and $\chi^\prime$. We denote
complex conjugation with superscript $^\ast$.
\mk{Peter's comment: What does the correlation between two epochs actually mean if they
are not on the backward light cone?}

The delta-function resolves one 3D Fourier
integral. We split the second integration into radial and spherical
coordinates, ${\rm d}^3 k = {\rm d} k k^2 {\rm d} \Omega_k$ and use once again
the orthogonality of the spherical harmonics to resolve the spherical integral. 
This leads to the potential power spectrum (\ref{eq:C_ell_phi_Pm}).




%%%%%%%%%%%%%%%%%%%%%%%%%%%%%%%%%%%%%%%%%%%%%%%%%%%%%%
\section{Discussion and comparison to previously published work}
%%%%%%%%%%%%%%%%%%%%%%%%%%%%%%%%%%%%%%%%%%%%%%%%%%%%%%

\paragraph{Kitching et al.~2016}

Eq.~\ref{eq:C_ell_full} is analogous to (7) and (8) from
\cite{2016arXiv161104954K} for a flat Universe and in the case of perfect
photometric redshifts, $p(z | z_{\rm p}) = \delta_{\rm D}(z - z_{\rm p})$, and
for a bin function that selects the redshift bin $i$, $W^{\rm SR}(z, z_{\rm p})
= 1_{z \in i-\mbox{th bin}}$. In \cite{2016arXiv161104954K} eq.~(7) is however the
factor $2/\pi$ missing.

In App.~A they derive the Limber approximation starting from the full spherical
projection. Note that their filter function $q$ defined in (31) has an
errornous factor of comoving distance $r$, and an additional factor $\pi/2$
appears.

Eq.~19 of \cite{2016arXiv161104954K} corresponds to our
(\ref{eq:P_ell_gamma}). \mk{Has to be checked.}

We can not reproduce the amplitude of the differences between the cases, neither for
the power spectrum nor for the shear correlation function (see Fig.~\ref{fig:xi_pm_K16}).


\paragraph{Bernardeau, Bonvin, Van de Rijt, Vernizzi (2012); van de Rijt (2012)}

Eq.~(44) of \cite{2012PhRvD..86b3001B} is the non-tomographic $C(\ell)$ in
terms of the time-independent potential power spectrum $T^2(k) P(k)$, where $T$
is the transfer function; the growth factor $D$ is contained in their window
function $W$ (43) as growth-suppression factor $g(\chi) = D(\chi)/a(\chi)$
(defined after eq.~{24}). For a single source redshift, we find the identity
$W(\chi)$ = $q(\chi)\chi^{-1} D(\chi) a^{-1}(\chi)$.
\cite{2012PhRvD..86b3001B} is consistent with our
(\ref{eq:C_ell_full}), if we interpret the
scale factors at distances $\chi$ and $\chi^\prime$, respectively.

The PhD thesis of Van de Rijt \cite{vande2012} presents an explicit calculation
of the second-order Limber approximation for a fixed source redshift. They
carry out the derivatives of the kernels $f$ under the assumption of a constant
growth-suppression factor $D_+(a)/a$. We show the consistency between their
result (eq.~7.19) and our (\ref{eq:C_ell_limber2_dr}), as follows.

We write (\ref{eq:C_ell_limber2_dr}) without tomography, and insert again the
time-independent power spectrum (\ref{eq:P_geom_mean}),
%
\begin{align}
  C^\gamma_{{\rm L2}}(\ell) = & - \frac 1 {\nu^2} \, \frac{\ellbar^2(\ell, 2)}{\nu^4} \, \pref^2
    \int {\rm d} \chi \, \chi^{2} P_{\rm m}\left(k = \frac\nu\chi\right)
    \nonumber \\
    & \times D_+(\chi) a^{-1}(\chi) \chi^{3/2} q(\chi) \left[ f^{\prime\prime} 
      + \frac{\chi}{3} f^{\prime\prime\prime} 
    \right](\chi)
  \label{eq:C_ell_limber2_dr_notomo}
\end{align}
%
For a fixed source comoving distance $\chi_{\rm S}$ the lensing efficiency is
$q(\chi) = (\chi_{\rm S} - \chi) \chi_{\rm S}^{-1}$. Assuming that
$D_+(\chi)/a(\chi) \approx g = \mbox{const}$, we get for the derivative terms,
with $f(\chi) = g \chi_{\rm S}^{-1} (\chi_{\rm S} - \chi) \chi^{-3/2}$,
%
\begin{equation}
  f^{\prime\prime}(\chi) + \frac{\chi}{3} f^{\prime\prime\prime}(\chi)
    = - \frac 1 8 \chi^{-5/2} \left( \frac 5 \chi
          + \frac 1 {\chi_{\rm S}} \right) \frac{D_+(\chi)}{a(\chi)}.
\end{equation}
%
Therefore,
%
\begin{align}
  C^\gamma_{{\rm L2}}(\ell) = \frac 1 {\nu^2} \, \frac{\ellbar^2(\ell, 2)}{\nu^4} \, \pref^2
    \int {\rm d} \chi \, \chi \, P_{\rm m}\left(k = \frac\nu\chi\right)
    \frac{q(\chi)}{8} \left( \frac 5 \chi + \frac 1 {\chi_{\rm S}} \right)
      \left(\frac{D_+(\chi)}{a(\chi)}\right)^2 
\end{align}
%
We now replace the matter with the potential power spectrum, ${\cal A}^2 P_{\rm m}(k) = k^{4} a^2 P_\Phi(k)$, and pull into the integral $\nu^{-4} = \chi^{-4} k^{-4}$. This yields
%
\begin{align}
  C^\gamma_{{\rm L2}}(\ell) = \frac 1 {\nu^2} \, \ellbar^2(\ell, 2)
    \int \frac{{\rm d} \chi}{\chi^2} \, P_\Phi\left(k = \frac\nu\chi\right)
    a^2(\chi) \frac{1}{8} \frac{\chi_{\rm S} - \chi}{\chi \chi_{\rm S}}
    \left( \frac 5 \chi + \frac 1 {\chi_{\rm S}} \right)
      \left(\frac{D_+(\chi)}{a(\chi)}\right)^2 .
\end{align}
%
Note the additional factor $a^2(\chi)$. This difference is accounted for by the
factor $a^{-1}$ in the propagation of primordial potential perturbations to
late times, see eq.~(6.22) in \cite{vande2012}, where the transition is done
using $g(a)$ instead of $D(a)$. \mk{Check this further.}


In Fig.~\ref{fig:L1L2E_Rijt} we reproduce Fig.~7.3 from \cite{vande2012} using
a similar set up (flat $\Lambda$ Universe with $\Omega_{\rm m} = 0.3, h=0.65,
\Omega_{\rm b} = 0.0461, \sigma_8 = 0.8, n_{\rm s} = 0.96$. All source galaxies
are at redshift $z_{\rm S} = 1$. The non-linear 3D matter power spectrum from
\cite{2012ApJ...761..152T} is used. The ratio of the first-and second- Limber approximated
power spectra to the full
projection shows excellent agreement at the sub-percent level.

\begin{figure}

  \begin{center}
    \resizebox{0.9\hsize}{!}{
      \includegraphics{P_kappa_limber_delta_Rijt}
    }
  \end{center}

    \caption{Differences in percentage of spherical shear power spectra terms
    as function of wave mode $\ell$, see Table \ref{tab:cases}.
    The blue and red curves corresponds to the ones shown in Fig.~7.3 of \citet{vande2012}.
    }

    \label{fig:L1L2E_Rijt}

\end{figure}



\paragraph{LoVerde \& Afshordi (2009)}

This paper introduces the extended Limber approximation to second order that we
apply in this work. Although they present the specific case of 2D galaxy
clustering, their calculations are general enough to apply to weak lensing.

Their eq.~(5) is a general spherical power spectrum of a scalar field,
projected from 3D to 2D via (4). Comparing their expressions with the
weak-lensing potential (\ref{eq:psi}), we can set $F_A(\chi) = F_B(\chi) = 2
c^{-2} D_+(\chi) q(\chi) \chi^{-1}$. With that, their eq.~(5) is identical to
our (\ref{eq:C_ell_phi_Pphi}).

The second-order Limber approximation in \cite{2008PhRvD..78l3506L} is presented in
eq.~(12). This is consistent with our first-order (\ref{eq:C_ell_limber1}) and
second-order (\ref{eq:C_ell_limber2_dr}) Limber approximation, when accounting for the difference between
lensing potential and shear 2D power spectrum, and 3D potential and matter power spectrum.


\paragraph{Schmidt (2009)}

In eq.~(9) \cite{2008PhRvD..78d3002S} write the lensing power spectrum in the
flat-sky limit. The Poisson factor $D_\phi$ is defined in their eq.~(5), see
also our eq.~\ref{eq:poisson}, and $D_{\rm m}$ is the growth factor. The filter
function $W_\kappa$ (eq.~10) can be written as
%
\begin{align}
  W_\kappa(z) = & \frac 1 {H(z)} \int_z^\infty {\rm d} z_{\rm s} W_{\rm L}(z_{\rm s}, z) W_{\rm g}(z)
              =  \frac 1 {H(z)} \chi \int_\chi^\infty {\rm d} \chi^\prime \frac{\chi^\prime - \chi}{\chi^\prime}
                n(\chi)
              \nonumber \\
              & \equiv \frac 1 {H(z)} \chi q(\chi),
\end{align}
%
using ${\rm d} z W_{\rm g}(z) = {\rm d} \chi n(\chi)$.
To write the redshift integrals (9) in \cite{2008PhRvD..78d3002S} over 
comoving distance, we make the variable transformation
%
\begin{equation}
  {\rm d} \chi = \frac{c {\rm d} z}{H(z)} .
\end{equation}
%
We recover our (\ref{eq:P_ell_gamma}).

\paragraph{Giannantonio et al. (2012)}

In eqs.~(25) and (26) \cite{2012MNRAS.422.2854G} write the flat-sky lensing
power spectrum. Their window function $W^{\varepsilon_i}$ defined in (25) for a
flat Universe equals $\pref \times q(\chi) / a(\chi)$, since ${\rm d} z {\rm d}
N(z)/{\rm d} z = {\rm d} \chi n(\chi)$; however there is a factor $r_K[r(z)]$
missing in the window function, which translates into a missing $r_K[r]
r_K[r^\prime]$ in the integral\footnote{Confirmed by T.~G., priv.~comm.}. This
also leads to an errorous $r_K^{-2}(r)$ in the Limber equation (27). With these
factors accounted for, we reproduce the expressions of \cite{2012MNRAS.422.2854G}.


\label{lastpage}

\end{appendix}

\end{document}

