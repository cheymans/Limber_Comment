In this paper we evaluate precision theoretical calculations for cosmic shear observables, bringing together different sources from the literature to provide a pedagogical review of the impact of adopting flat-sky and Limber approximations.  We demonstrate that for current surveys, such as CFHTLenS, these approximations have a negligible impact on cosmological parameter constraints.

For future surveys, the decrease in statistical errors places higher requirements on the accuracy of the theoretical modelling.    There is also, however, the need to be able to rapidly sample the multi-dimensional cosmological parameter likelihoods.  This requirement for computational speed is incompatible with a theoretical analysis that calculates a full spherical solution for the shear power spectrum, without adopting any approximation.  We therefore present alternative solutions, revisiting the work of  \citet{2012PhRvD..86b3001B} who showed that adopting the second-order extended Limber approximation of \citet{2008PhRvD..78l3506L} provides a representation of the full spherical solution for the shear power spectrum that is accurate at the sub-percent level for $\ell > 3$.    We have verified this result and provide to the community our fast numerical implementation of all the approximations studied in this analysis,
and the slow calculation of the full projection within the publicly available package \textsc{nicaea} at \texttt{http://www.cosmostat.org/software/nicaea}.

Finally we propose that future surveys seek to optimise the statistical analyses of their cosmic shear data.  For example moving from the standard two-point shear correlation function statistic to the more stringent `COSEBI' statistic \citep{COSEBIs} renders the cosmic shear measurement insensitive to the low-$\ell$ scales where the Limber and flat-sky approximations have an impact on the precision of the theoretical modelling.  
