%\documentclass[a4paper,preprint, fleqn]{aastex}
%\documentclass[a4paper, preprint, fleqn, usenatbib]{aastex}
\documentclass[fleqn, usenatbib]{mnras}
\usepackage{graphicx}   % Including figure files
\usepackage{amsmath}	% Advanced maths commands
\usepackage{amssymb}	% Extra maths symbols
\usepackage{color,verbatim,url}
\usepackage{tabularx}
\usepackage[table,usenames,dvipsnames]{xcolor}
\usepackage{pdflscape}
\usepackage{lastpage}
\usepackage{newtxtext, newtxmath}

\usepackage[T1]{fontenc}
\usepackage{ae, aecompl}

\definecolor{pink}{rgb}{0.858, 0.188, 0.478}
\definecolor{purple}{RGB}{76, 0,153}

\newcommand{\red}[1]{{\color{red}{#1}}}
\newcommand{\green}[1]{{\color{green}{#1}}}
\newcommand{\blue}[1]{{\color{blue}{#1}}}
\newcommand{\x}{\vec{x}}
\renewcommand{\d}[0]{{\rm d}}
\newcommand{\shahab}[1]{\textcolor{pink}{\bf #1}}

\newcommand{\be}{\begin{equation}}  \newcommand{\ee}{\end{equation}}
\newcommand{\mb}[1]{\mbox{ #1 }}  
\newcommand{\ba}{\begin{eqnarray}}\newcommand{\ea}{\end{eqnarray}}
\newcommand{\bm}[1]{\mbox{\boldmath{$#1$}}}   %this is bold italic for MNRAS
%\newcommand{\mat}[1]{\mathbfss{#1}}
\newcommand{\diff}{\mathrm{d}}

\title[Comment on `The Limits of Cosmic Shear']{Comment on `The Limits of Cosmic Shear'}

\author[M.~Kilbinger et~al.]
{Martin Kilbinger$^{1,2}$\thanks{E-mail: martin.kilbinger@cea.fr},
Catherine Heymans$^3$,
Marika Asgari$^3$, 
Shahab Joudaki$^{4,5}$, 
\newauthor
Peter Schneider$^6$,
Patrick Simon$^6$,
Ludovic Van Waerbeke$^8$,
Joachim Harnois-D\'eraps$^3$,
\newauthor
Hendrik Hildebrandt$^6$,
Fabian K\"ohlinger$^7$,
Konrad Kuijken$^7$, and
Massimo Viola$^7$
\\
$^1$CEA/Irfu/SAp Saclay, Laboratoire AIM, 91191 Gif-sur-Yvette, France\\
$^2$Institut d'Astrophysique de Paris, UMR7095 CNRS, Universit\'e Pierre \& Marie Curie, 98 bis boulevard Arago, 75014 Paris, France \\
$^3$Institute for Astronomy, University of Edinburgh, Royal Observatory, Blackford Hill, Edinburgh EH9 3HJ, UK\\
$^4$Centre for Astrophysics \& Supercomputing, Swinburne University of Technology, PO Box 218, Hawthorn, VIC 3122, Australia\\
$^5$ARC Centre of Excellence for All-sky Astrophysics (CAASTRO)\\
$^6$Argelander-Institut f\"ur Astronomie, Auf dem H\"ugel 71, 53121 Bonn, Germany\\
$^7$Kavli Institute for the Physics and Mathematics of the Universe (WPI), The University of Tokyo Institutes for Advanced Study, \\
The University of Tokyo, Kashiwa, Chiba 277-8583, Japan\\
$^8$Department of Physics and Astronomy, University of British Columbia, 6224 Agricultural Road, Vancouver, BC V6T 1Z1, Canada\\
}




\date{Released 16/12/2016}

\pagerange{\pageref{firstpage}--\pageref{LastPage}} \pubyear{2016}

\begin{document}
\setlength{\voffset}{-12mm}

\label{firstpage}

\maketitle

\begin{abstract}

%\author{Members of the KiDS, CFHTLenS and RCSLenS collaborations; \\
%J. Harnois-D\'eraps, C. Heymans, H. Hildebrandt, S. Joudaki,  M. Kilbinger, F. K\"ohlinger, K. Kuijken, P. Schneider, P. Simon, L. Van Waerbeke, M. Viola}%

Kitching et al. (2016) examine the impact of the flat-sky and Limber approximations that can be used when constraining cosmological parameters from the measurement of weak gravitational lensing by large-scale structures, also known as cosmic shear.  We support the core message of this paper that the use of such approximations should be considered in the future analyses of high precision surveys such as Euclid, LSST and WFIRST.  In this comment, we argue that these approximations have negligible impact for current surveys and are thus unable to significantly reduce the tension between the constraints from some current weak lensing surveys and the Planck CMB experiment.  Reducing uncertainty and characterising bias in photometric redshift estimation is a critical task for the success of current and future surveys.  We highlight that sophisticated treatments of these uncertainties have already been explored in the literature and have been shown not to resolve the tension with Planck, in contrast to the message that Kitching et al. (2016) convey.

\end{abstract}

\begin{keywords}
cosmology: observations -- gravitational lensing: weak -- galaxies: photometry -- surveys
\end{keywords}


\section{Introduction}

The measurement of weak gravitational lensing by large-scale structures provides a powerful cosmological probe of dark matter, dark energy, and modifications to gravity.  As such it is the primary science goal of several current (KiDS, HSC, DES) and future (Euclid, LSST, WFIRST) large-scale surveys. Interest in the results from these surveys is high as statistically significant deviations have been found between the cosmological parameter constraints from the CMB Planck experiment \citep{planck/cosmo:2015} in comparison to weak lensing constraints from both the Kilo-Degree Survey \citep[KiDS;][]{hildebrandt/etal:2016} and the Canada-France-Hawaii Telescope Lensing Survey \citep[CFHTLenS;][] {joudaki/etal:2016}.  If the source of this tension is not a result of so-far unconsidered sources of systematic errors in one or all experiments, extensions to the standard flat $\Lambda$CDM cosmological models need to be considered.  \citet{joudaki/etal:2017} have shown, for example, that the tension can be resolved with an evolving dark energy model.

In this comment paper we discuss the flat-sky and Limber approximations that can be used when deriving theoretical predictions for the principally used cosmic shear statistic, $\xi_\pm$, the two-point shear correlation functions.  \citet{kitching/etal:2016} report that when considering the flat-sky and Limber approximations in concert, tensions between weak lensing and CMB experiments can be reduced. They illustrate the effect with a re-analysis of the non-tomographic weak lensing data from the CFHTLenS survey \citep{kilbinger/etal:2013}.    On initial inspection we found that these claims appeared implausible, for two reasons: First, in a naive estimate, the deviation of a flat-sky approximation from a full treatment is expected to be of a similar order to the fractional difference between $\theta_{\rm max}$ and $\sin\theta_{\rm max}$,
where $\theta_{\rm max}$ is the maximum angular scale considered in the cosmic shear analysis. In the case of \citet{hildebrandt/etal:2016}, where $\theta_{\rm max}=50$ arcmin, $(\theta_{\rm max} - \sin\theta_{\rm max})/\theta_{\rm max} \approx 3.5 \times 10^{-5}$.  Second, it was shown in \citet[][see their Fig.3]{giannantonio/etal:2012} that the difference between the Limber approximated shear power spectrum and one calculated using an exact treatment is far smaller than cosmic variance, even for a full-sky cosmic shear survey.

This comment seeks to explore the content of \citet{kitching/etal:2016} in detail, discussing why we disagree with their findings.  It is structured as follows.  In section~\ref{sec:theory} we clarify which brand of approximations, highlighted by \citet{kitching/etal:2016}, have been used in recent weak lensing analyses \citep{joudaki/etal:2016, hildebrandt/etal:2016, joudaki/etal:2017}.  We are unable to replicate key results from \citet{kitching/etal:2016} and show that we do not expect these approximations to impact on the cosmological analysis of current surveys.  In section~\ref{sec:cfhtlens} we present a `one-parameter' cosmological re-analysis of 2D cosmic shear data from \citet{kilbinger/etal:2013} which is inconsistent with the analysis presented in \citet{kitching/etal:2016}.  We also briefly review the literature that has advanced our understanding since the original CFHTLenS analyses were published and discuss the impact of flat-sky and Limber approximations in our analysis of numerical simulations, CMB lensing and quadratic estimators for the cosmic shear power spectrum.

In section~\ref{sec:photoz} we review our current knowledge of photometric redshift uncertainties and the sophisticated mitigation strategies that have already been applied to data.  We suggest that a possible misinterpretation in the sign of the redshift biases reported in \citet{choi/etal:2016} has led to the incorrect conclusion of \citet{kitching/etal:2016} that these redshift biases can resolve the tension with Planck.  

\section{Theory}
\label{sec:theory}
\subsection{The Flat-Sky approximation and Extended Limber Approximation}
The baseline measurement for many cosmic shear studies has focused on the two-point shear correlation function between two tomographic bins $ij$ \citep[for more details see][and references therein]{miraldaescude:1991, kaiser:1992, bartelmann/schneider:2001}, given by
\be
\xi_\pm^{ij}(\theta) = \frac{1}{2\pi}\int_0^\infty \d\ell \,\ell \,P^{ij}_\gamma(\ell) \, J_{0,4}(\ell \theta) \, , 
\label{eqn:xiGG}
\ee
where $J_{0,4} (\ell \theta)$ is the zeroth (for $\xi_+$) or fourth (for $\xi_- $) order Bessel function of the first kind and $P_\gamma(\ell)$ is the shear power spectrum at angular wave number $\ell$ which can be related to the underlying matter power spectrum $P_\delta$, using a Limber approximation \citep{limber:1953} as,
\be 
P^{ij}_\gamma(\ell) = T_l \int_0^{\chi_{\rm H}} \d \chi \, \frac{q_i(\chi)q_j(\chi)}{[f_K(\chi)]^2} \, P_\delta \left( \frac{\nu(\ell)}{f_K(\chi)},\chi \right).
\label{eqn:Pkappa} 
\ee
Here $f_K(\chi)$ is the comoving angular diameter distance out to comoving radial distance $\chi$ and $\chi_{\rm H}$ is the comoving horizon distance.   The lensing efficiency function $q(\chi)$ is given in equation (5) of \citet{hildebrandt/etal:2016}, and the value of $\nu(\ell)$ depends on whether a baseline Limber approximation, $\nu(\ell) = \ell$, or the more accurate extended Limber approximation, $\nu(\ell) = \ell + 0.5$, from \citet{loverde/afshordi:2008} is used.   \citet{kitching/etal:2016} show that the pre-factor $T_\ell$ for the shear power spectrum is given by
\be
T_\ell = \frac{(\ell+2)(\ell+1)\ell(\ell-1)}{(\ell + 0.5)^4} = 1-{5\over 2\ell^2} +{\cal O}(\ell^{-3}) \, .
\label{eqn:Tl}
\ee
The corresponding pre-factor for the convergence power spectrum, along with the pre-factors for a range of other statistics such as the CMB lensing power spectrum and galaxy-galaxy lensing power spectrum, is given in \citet{jk12}.

When a survey is sufficiently small, a flat-sky approximation can be made reducing $T_\ell =1$ and $\nu(\ell) = \ell$.  In \citet{kitching/etal:2016} it is not fully clear which combinations of  $T_\ell$ and $\nu(\ell)$ are considered, so we review them all as summarised in Table~\ref{tab:Tl_nu}, also highlighting which combinations were used in recent cosmic shear analyses.

 \begin{table}[htb]
\begin{center}
\begin{tabular}{ | l | l | c | c  | c |}
\hline
Case & ID & $T_\ell$ & $\nu(\ell)$ & Used in \\ \hline
\citet{kitching/etal:2016} `Standard' & KSt & $1$ & $\ell$ & pre-2014 CFHTLenS papers \\
Baseline Limber `Flat' Sky &  LF & $\ell^4 / (\ell + 0.5)^4$ & $\ell$ & \\
Baseline Limber Spherical & LS & equation~\ref{eqn:Tl} & $\ell$ & \\
Extended Limber `Flat' Sky & ELF & $\ell^4 / (\ell + 0.5)^4$ & $\ell + 0.5$ & \\
Extended Limber Spherical & ELS & equation~\ref{eqn:Tl}& $\ell + 0.5$  & \\
Extended Standard Flat Sky & ESt & $1$ & $\ell + 0.5$ & \citet{joudaki/etal:2016} \&  \\
  &  & & & \citet{hildebrandt/etal:2016}$^*$\\\hline
 \end{tabular}
 \end{center}
 \caption{\label{tab:Tl_nu}Variations on the different approximations that can be made when calculating the shear power spectrum $P_\gamma(\ell)$ (equation~\ref{eqn:Pkappa}), using a Limber approximation.  $^*$We confirm that there is a typographical error in equation 4 of \citet{hildebrandt/etal:2016} which should include the extra term of `$+0.5$' in $\nu(\ell)$ that was incorporated in the cosmological analysis.} 
 \end{table}

Figure~\ref{fig:Cl_xi} shows theoretical models for the shear power spectrum $P_\gamma(\ell)$ (equation~\ref{eqn:Pkappa}), and the shear correlation function (equation~\ref{eqn:xiGG}) for the different variations on the approximations listed in Table~\ref{tab:Tl_nu}.   The Standard and Spherical models agree at the few percent level above $\ell>10$ and below $\theta< 100$ arcmin.  The two outliers are the Baseline and Extended `Flat' Sky Limber cases (LF and ELF) where the \citet{kitching/etal:2016} `Flat' Sky approximation has been made with
\be
T_\ell^{\rm `Flat'} = \ell^4 / (\ell + 0.5)^4 = 1-{2\over \ell}+{5\over 2\ell^2} +{\cal O}(\ell^{-3}) \, . 
\ee    
We argue that applying the approximation, that $\ell \pm x \approx \ell$, where $x$ is small, only to the numerator of equation~\ref{eqn:Tl} is inconsistent.  By comparing the expansions of equations~\ref{eqn:Tl} and~\ref{eqn:Tlflat}, we see that the \citet{kitching/etal:2016} `Flat' Sky approximation effectively introduces a new term $2/\ell$ making it more deviant from the full spherical solution in comparison to the standard $T_\ell = 1$ case.    To our knowledge, this form of the flat-sky approximation has not been used in cosmic shear studies to date.  

As shown in the lower panels of Figure~\ref{fig:Cl_xi}, the `standard flat' sky approximation used by recent cosmic shear surveys\footnote{The KSt approximation was used for pre-2014 CFHTLenS analysis and the ESt approximation was used for recent CFHTLenS and KiDS analyses}, where $T_\ell = 1$, recovers a very similar result to the spherical sky case.  Over angular scales used in the recent KiDS cosmic shear analysis (where the maximum angular scale $\theta<50.7$ arcmin is shown dashed), these curves differ by less than 1\%.  As such the standard flat-sky approximations that have been used will not impact the cosmological analyses of current surveys.
 
 \begin{figure}
 \begin{center}
 \includegraphics[width=0.85\textwidth]{figures/Cl_xi_comp.pdf}
 \caption{ \label{fig:Cl_xi}\emph{Upper Panels:} Theoretical models for the shear power spectrum (\emph{left}) and shear correlation function $\xi_+$ (\emph{right}) for the different approximations listed in Table~\ref{tab:Tl_nu}, assuming a \citet{planck/cosmo:2015} cosmology and the 2D CFHTLenS redshift distribution. \emph{Lower Panels:} Deviations with respect to the extended Limber spherical sky model (ELS).    The dashed lines in the right panels show the largest angular scale used in the recent KIDS cosmic shear analysis where an Extended Standard (ESt) analysis was used.}
 \end{center}
 \end{figure}
 
 \subsection{Cosmic shear without the Limber Approximation}
In collecting our comments on \citet{kitching/etal:2016} we contacted colleagues who had previously tested the impact of the Limber approximation on cosmic shear studies by comparing a Limber-approximated cosmic shear power spectrum with an exact calculation.  The majority of these analyses, carried out many years ago, remained unpublished in peer reviewed journals as no significant deviation was found\footnote{See for example appendix M of the 2010 PhD thesis from Donghui Jeong (\url{http://www.personal.psu.edu/duj13/dissertation/djeong_diss_appM.pdf}) and Chapter 7 of 2012 PhD thesis from Nicolas Van de Rijt (\url{http://ipht.cea.fr/Docspht/articles/t12/080/public/these_vanderijt2012.pdf}).}.  One notable exception, however, is \citet{giannantonio/etal:2012} who find that for a cosmic shear power spectrum, the extended Limber approximation is consistent with the exact calculation for $\ell>5$, with any differences well within cosmic-variance errors for a Euclid-like survey\footnote{\citet{kitching/etal:2016} comment on this result, but indicate that it is in error owing to the application of a fixed low-$\ell$ limit.  \citet{giannantonio/etal:2012} however state that this limit is only applied in a positive curvature case.}.  They conclude that the Limber approximation works well for cosmic shear owing to the wide redshift distribution of the lensed sources.  This is in comparison to the case of galaxy power spectra,  where narrower redshift bins result in a significant difference between the Limber and non-Limber cases \citep[see also][]{simon/2007}.   Given this body of work, we argue that the assertion by \citet{kitching/etal:2016} that the majority of cosmic shear data analyses to date have made an `axiomatic assumption (i.e. an unquestioned and untested assumption at the beginning of the analysis)'  is unwarranted.  Not only is there published work that has questioned and tested the Limber approximation for the cosmic shear power spectrum, individual teams have also verified this result internally for their own cosmic shear analyses.    

Given the insignificant effect of using the Limber approximation found by various groups, the ability to calculate an exact non-Limber solution has unfortunately not been maintained in many theoretical codes over the years.  Instead these codes have evolved to account for critical approximations such as the impact of baryon feedback on the matter power spectrum and non-zero neutrino masses \citep[see for example][]{joudaki/etal:2016, mead/etal:2016}.    Two exceptions to this are recent updates to CLASS \citep{blas/lesgourgues/tram:2011,audren/etal:2013} and CosmicFish \citep{raveri/etal:2016}.  On-going code comparisons have, however, revealed differences that are larger than the expected impact of Limber or non-Limber approximation.   We are therefore unable to readily determine non-Limber theoretical calculations for the shear-correlation function to directly compare with the results of \citet{kitching/etal:2016}.   Instead we refer the reader to Figure 3 of \citet{giannantonio/etal:2012} to draw their own conclusions about the impact of this effect.



 
 
 
 
 
 
 

\section{Advances in understanding since CFHTLenS}
\label{sec:cfhtlens}
%\subsubsection{CFHTLenS updates and advances since 2014}

The Canada-France-Hawaii Telescope Lensing Survey (CFHTLenS) represented a
major step forward for the field of weak gravitational lensing, in terms of
improved accuracy in data reduction \citep{CFHTLenS-data}, the implementation
of PSF-Gaussianised matched multi-band photometry
\citep{CFHTLenS-photoz}, cross-correlation clustering analysis between
photometric redshift slices to verify tomographic redshift distributions
\citep{CFHTLenS-2pt-tomo}, accurate calibrated shape measurements
\citep{CFHTLenS-shapes} and a full suite of informative systematic tests to
select a clean data set \citep{CFHTLenS-sys}. Since the public release
of this survey in 2013, the community has continued to scrutinise and advance
our understanding of CFHTLenS by identifying a number of areas where analyses
could improve:
%
\begin{itemize}
%
 \item{\citet{2016MNRAS.463.3737C} identified significant biases in the tomographic
photometric redshift distributions using a more effective clustering analysis,
in comparison to \citet{CFHTLenS-2pt-tomo}, by incorporating newly overlapping
spectroscopic data from the Sloan Digital Sky Survey.  The conclusion of this
work was that any re-analysis of CFHTLenS should include systematic error terms
to account for bias and scatter, with a prediction that accounting for these
biases would {\it reduce} the recovered amplitude of $\sigma_8$ by $\sim
4$\%. Additional new techniques to calibrate the redshift distribution of tomographic
bins was introduced recently in \cite{KiDS-450}.}
%
\item{The CFHTLenS tomographic cosmological analysis was then revisited by
\citet{joudaki/etal:2016} in order to include a full redshift error analysis
based on the results from \citet{2016MNRAS.463.3737C}.  The impact of
correcting for these biases, including their associated errors, served to
reduce the overall constraining power of the survey and hence also the tension
between CFHTLenS and CMB constraints.}
%
 \item{\cite{asgari/etal:2017} used the stringent COSEBI statistic
\citep{COSEBIs} to identify significant non-lensing B-mode distortions when the
CFHTLenS data was split into tomographic slices.}
%
\item{\citet{2015MNRAS.454.3500K} showed that the CFHTLenS shear calibration
corrections derived in \citet{CFHTLenS-shapes} were underestimated as a result
of an imperfect match between the galaxy populations in the data and image
simulations.}
%
\item{\citet{2016arXiv160605337F} demonstrated that the CFHTLenS data would
have been subject to a weight bias that favours galaxies that are more
intrinsically oriented with the point-spread function.  They also showed that
the impact of calibration selection biases, that were not considered in
\citet{CFHTLenS-shapes}, would have lead to the over-correction of
multiplicative shear bias in the CFHTLenS analyses, by a few percent.}
%
\item{\citet{joudaki/etal:2016} updated the CFHTLenS covariance matrices using
larger-box numerical simulations that were less subject to the lack of power on
large scales. A complementary accurate estimate of the covariance matrix using
analytical methods will be published soon (Joachimi et al.~in prep.}
%
\item{\cite{2012ApJ...761..152T} provided a more accurate non-linear power
spectrum correction than that used in the original CFHTLenS analyses, and the
halo model from \cite{2015MNRAS.454.1958M} allowed for simultaneous modelling
of baryonic modifications to the non-linear power spectrum.} 
%
\end{itemize}
%
All these advances in our understanding were incorporated and accounted for in
the recent KiDS cosmic shear analysis \citep{KiDS-450} which reports a $2.3
\sigma$ tension with Planck.  Efforts are now underway to fully re-analyse
CFHTLenS using the advanced KiDS analysis pipeline with revised shape
measurements and calibrations for the shear and photometric redshifts. Until
this analysis is complete we note that these known shortcomings with the
original CFHTLenS results impact in different ways the cosmological conclusions
that one can draw. As CFHTLenS has similar statistical power
to current weak lensing surveys, however, it nevertheless provides a very
useful testbed with which to demonstrate the impact of adopting different
approximations when constraining cosmological parameters.

%\subsubsection{Cosmological analysis setup}

In this work, we focus on the weak-lensing power spectrum projection, and
assess the impact of various approximations on cosmological constraints from
CFHTLenS. For consistency with the original analysis
\citep{CFHTLenS-2pt-notomo}, we adopt the same priors and non-linear power
spectrum corrections from \cite{2003MNRAS.341.1311S}.

We re-analyse the 2D CFHTLenS measurement of the two-point shear correlation
function $\xi_\pm(\theta)$ from \cite{CFHTLenS-2pt-notomo}, defined in
equation~(\ref{eqn:xiGG}). As in \cite{CFHTLenS-2pt-notomo} we fit both
components $\xi_+$ and $\xi_-$ between angular scales $\theta = 0.8$ and $350$
arc minutes, and use a $N$-body simulation estimate of the non-Gaussian
covariance including the cross-covariance between both components. Bayesian
Population Monte-Carlo parameter sampling is performed using the publicly
available software
\textsc{CosmoPMC}\footnote{\texttt{http://www.cosmostat.org/software/cosmopmc}}
\citep{WK09,KWR10}. The cosmological modelling part includes the various
lensing projections, calculated using the software library
\textsc{nicaea}\footnote{\texttt{http://www.cosmostat.org/software/nicaea}}.

%%%%%%%%%%%%%%%%%%%%%%%%%%%%%%%%%%%%%%%%%%%%%%%%%%%%%%
%\subsubsection{Cosmological parameter results}
%\label{ref:cosmo_results}
%%%%%%%%%%%%%%%%%%%%%%%%%%%%%%%%%%%%%%%%%%%%%%%%%%%%%%


For a first-order standard Limber flat-sky approximation (L1Fl) we find
$\sigma_8 (\Omega_{\rm m}/0.27)^{0.6} =0.787^{+0.031}_{-0.033}$, the same
result that was published in \cite{CFHTLenS-2pt-notomo}. Using the second-order
extended Limber flat-sky hybrid approximation (ExtL2FlHyb) results in $\sigma_8
(\Omega_{\rm m}/0.27)^{0.6} = 0.788 \pm 0.032$, a negligible change of the
amplitude that is well within the Monte-Carlo sampling noise. The largest
difference is measured with the depreciated case ExtL1Fl, for which the
recovered amplitude is larger by $16\%$ of the statistical error. \ch{These
negligible changes to the error bars were to be expected owing to the
high level of statistical noise and cosmic variance in comparison to the low-level impact of the various
approximations shown in Fig.~\ref{fig:Cl_cases}.}

Table \ref{tab:CFHTLenS_Sigma8} lists the mean and 68\% credible interval for
$\sigma_8 \Omega_{\rm m}^{0.6}$ for the various approximations to the
lensing power spectrum projections listed in Table~\ref{tab:cases} Note again
that these values do not represent the state-of-the-art cosmological results,
since many of the above listed analysis advancements made since 2013 have not
been taken into account. As an example of a significant effect, when using the
revised non-linear power spectrum of \cite{2012ApJ...761..152T} in place of
\cite{2003MNRAS.341.1311S}, there is a decrease of $0.6 \sigma$ with $\sigma_8
(\Omega_{\rm m}/0.27)^{0.6} =0.768^{+0.029}_{-0.031}$.

\ch{Considering the cosmological constraints from tomographic Kilo-Degree Survey (KiDS), we conclude that
these are robust to flat-sky and Limber approximations. The case ExtL1FlHyb that was used
for the analysis of KiDS data in \citet{KiDS-450} and \cite{joudaki/etal:2017} introduces
errors that are more than an order of magnitude lower than the cosmic variance
for that survey, and thus this approximation has a negligible impact on the
cosmological parameters.}



\renewcommand{\baselinestretch}{1.5}
\begin{table}
\begin{centering}
  
  \caption{\label{tab:CFHTLenS_Sigma8}Mean and 68\% confidence interval for 
  $\sigma_8 (\Omega_{\rm m}/0.27)^{0.6}$ and $\sigma_8 (\Omega_{\rm m}/0.3)^{0.6}$
  for various approximations to the lensing
  power spectrum projections listed in Table~\ref{tab:cases}.}

  \begin{tabular}{lcc} \hline
  ID         & $\sigma_8 (\Omega_{\rm m}/0.27)^{0.6}$ & $\sigma_8 (\Omega_{\rm m}/0.3)^{0.6}$ \\ \hline
  L1Fl       & $0.787^{+0.031}_{-0.033}$ & $0.739^{+0.029}_{-0.031}$ \\
  ExtL1Fl    & $0.792 \pm 0.032$ & $0.744 \pm 0.030$ \\
  ExtL1FlHyb & $0.788^{+0.031}_{-0.033}$ & $0.740^{+0.029}_{-0.031}$ \\
  ExtL2FlHyb & $0.788^{+0.031}_{-0.033}$ & $0.740^{+0.029}_{-0.031}$ \\
  ExtL2Sph   & $0.789^{+0.031}_{-0.032}$ & $0.740^{+0.029}_{-0.030}$ \\ \hline
  \end{tabular}

\end{centering}
\end{table}
\renewcommand{\baselinestretch}{1}




\section{Photometric redshift bias and uncertainty}
\label{sec:photoz}
In \citet{choi/etal:2016} a cross-correlation clustering analysis, between photometric redshift slices and overlapping spectroscopic redshifts, is used to determine linear biases in the CFHTLenS photometric redshift distributions.   The maximum redshift $z_B<0.9$ in this analysis was limited by the redshift overlap with the spectroscopic sample\footnote{The implication of a $z_B < 0.9$ high-redshift limit is that the analysis is unable to constrain the high-redshift tails of the redshift distributions due to the lack of overlapping high-redshift spectroscopic samples. As such the results from \citet{choi/etal:2016} are only the first important step towards a detailed understanding of the CFHTLenS tomographic redshift distributions and as such may be revised in the future.}.  \citet{choi/etal:2016} found significant biases in the redshift distributions used in the original CFHTLenS analyses,
concluding that any re-analysis of CFHTLenS should include systematic error terms to account for bias and scatter.    This re-analysis was presented in \citet{joudaki/etal:2016} where the impact of correcting for the biases determined by \citet{choi/etal:2016}, including their associated errors, served to reduce the overall constraining power of the survey and hence also the tension between constraints.  The best-fit model, however moved further from the Planck-preferred model, in agreement with the `toy-model' analysis of \citet{choi/etal:2016} which concluded that the effect of these biases would be to {\it reduce} the recovered amplitude of $\sigma_8$ by $\sim 4$\%. 

\citet{choi/etal:2016} additionally showed that applying a simple linear shift to the photometric redshift distributions was insufficient when modelling the true underlying redshift distribution.  In \citet{hildebrandt/etal:2016}, a direct calibration method was used to determine the redshift distribution of four tomographic bins.  Multiple bootstrap samples of the resulting calibrated distribution allowed for the characterisation of both the uncertainty in the mean and shape of the redshift distribution.  The inclusion of this more sophisticated treatment of photometric redshift errors was not found to alleviate the tension between Planck and KiDS parameter constraints.

\cite{kitching/etal:2016} use a linear fit to the biases determined by \citet{choi/etal:2016} in order to extrapolate the results beyond the photometric redshift limit of $z_B<0.9$ and determine a bias at each redshift.  They then correct the non-tomographic \citet{kilbinger/etal:2013} redshift distribution with this bias relation.  They report that this correction results in an {\it increase} to the recovered amplitude of $\sigma_8$ by $\sim 4$\%, which resolves the tension with Planck.   We are unable to recover this conclusion using two different methods.   Based on the \cite{kitching/etal:2016} relation, the predicted shift at the mean CFHTLenS redshift $z=0.747$ is $\Delta_z = 0.056$, where we use the convention from \citet{choi/etal:2016} that a positive $\Delta_z$ serves to increase the true mean redshift.  Using the same relation to individually shift each component of the full \citet{kilbinger/etal:2013} redshift histogram also results in a change to the mean redshift of $\Delta_z = 0.056$.  Increasing the mean redshift implies a reduction in the recovered amplitude of $\sigma_8$.  Using the same `toy-model' analysis of \citet{choi/etal:2016} we would expect this change in the mean redshift to {\it reduce} the recovered amplitude of $\sigma_8$ by $\sim 5$\%.    From this we suggest that \cite{kitching/etal:2016} could have misinterpreted the direction of the redshift biases reported in \citet{choi/etal:2016}. 



 

\section{Conclusion}
\label{sec:conclusion}
In this comment we have highlighted a number of shortcomings in the analysis presented in \cite{kitching/etal:2016}.  We argue that the flat sky and Limber approximations that can be used in cosmic shear analyses are not able to explain away the tension between current weak lensing and Planck results, in contrast to the message that \cite{kitching/etal:2016} convey.  Our critique should not, however, detract from the extremely important core message of their paper, that future surveys will need to carefully consider these approximations and optimise their statistical analyses accordingly.  For example moving from the standard two-point shear correlation function statistic to the more stringent `COSEBI' statistic renders the cosmic shear measurement completely insensitive to the low-$\ell$ scales where these approximations become important \citep{schneider/etal:2010}.  


\bibliographystyle{apj}
\bibliography{Limber_Comment}
\end{document}
