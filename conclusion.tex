In this comment we have highlighted a number of shortcomings in the analysis presented in \cite{kitching/etal:2016}.  We argue that the flat-sky and Limber approximations that can be used in cosmic shear analyses are not able to explain away the tension between current weak lensing and Planck results, in contrast to the message that \cite{kitching/etal:2016} convey.  Our critique should not, however, detract from the important core message of their paper, that future surveys will need to consider these approximations and optimise their statistical analyses accordingly.  For example moving from the standard two-point shear correlation function statistic to the more stringent `COSEBI' statistic \citep{schneider/etal:2010} renders the cosmic shear measurement insensitive to the low-$\ell$ scales where these approximations become important.  

The treatment of photo-$z$ uncertainties in \cite{kitching/etal:2016} is based on a simple linear shift of redshift distributions. This has been shown to be insufficient to capture the full complexity of the problem.  More importantly though, we suggest that \cite{kitching/etal:2016} have applied the photo-$z$ calibrations from \cite{choi/etal:2016} in the wrong sense. Instead of alleviating the tension between CFHTLenS and Planck, a correct application of the offsets would increase the discrepancy, as shown with a recent re-analysis of the CFHTLenS data \citep{joudaki/etal:2016} as well as results from the KiDS project \citep{hildebrandt/etal:2016}.

\section{Acknowledgements}
We thank Sarah Bridle, Ami Choi, Tommaso Giannantonio, Donnacha Kirk, Lance Miller, Benjamin Joachimi, Chris Blake and Adam Amara for useful discussions during the collection of these comments.