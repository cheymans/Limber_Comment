The Canada-France-Hawaii Telescope Lensing Survey (CFHTLenS) represented a major step forward for the field of weak gravitational lensing, in terms of improved accuracy in data reduction \citep{erben/etal:2013}, the implementation of gaussianised matched multi-band photometry \citep{hildebrandt/etal:2012}, cross-correlation clustering analysis between photometric redshift slices to verify tomographic redshift distributions \citep{benjamin/etal:2013}, accurate shape measurements \citep{miller/etal:2013} and a full suite of informative systematic tests to select a clean data set \citep{heymans/etal:2012}.    Since the public release of this survey in 2013, the community has continued to scrutinise and advance our understanding of CFHTLenS by identifying a number of areas where the analysis could improve;
\begin{itemize}
\item{\citet{choi/etal:2016} identified biases in the tomographic photometric redshift distributions using a more effective clustering analysis, in comparison to \citet{benjamin/etal:2013}, by incorporating newly overlapping spectroscopy from the Sloan Digital Sky Survey.}
\item{\citet{asgari/etal:2016} used the stringent `COSEBI' statistic to identify significant non-lensing `B-mode' distortions when the CFHTLenS data was split into tomographic slices.}
\item{\citet{kuijken/etal:2015} showed that the CFHTLenS shear calibration corrections derived in \citet{miller/etal:2013} were underestimated as a result an imperfect match between the galaxy populations in the data and image simulations.}
\item{\citet{fenechconti/etal:2016} demonstrated that the CFHTLenS data would have been subject to a weight bias that favours galaxies that are more closely oriented with the point-spread function.}
\item{\citet{joudaki/etal:2016} updated the CFHTLenS covariance matrices using larger-box numerical simulations that were less subject to the lack power on large scales.}
\end{itemize}
Given this well publicised list of shortcomings with the original CFHTLenS results, it was surprising that \citet{kitching/etal:2016} chose to focus their study on the now outdated cosmic shear analysis of \citet{kilbinger/etal:2013}.  Furthermore, out of all the CFHTLenS cosmic shear results, the non-tomographic analysis of \citet{kilbinger/etal:2013} is the least in tension with Planck \citep[see for example][]{fu/etal:2014,abbott/etal:2016}, with $\sigma_8 (\Omega_m/0.3)^{0.5} = 0.738^{+0.055}_{-0.032}$, an agreement with Planck at $1.6 \sigma$. 

\citet{kitching/etal:2016} choose to only vary $\sigma_8$ in their analysis, fixing all other parameters to Planck constraints.  They find a conditional value of $\sigma_8 = 0.789 \pm 0.015$ for their `standard analysis' which we review in Section~\ref{sec:theory}.    We argue that this artificial reduction in the error could easily mislead as it makes it appear that the impact of removing all approximations from the theoretical analysis is more significant for current surveys than it is in reality.

We conclude this section by noting that all the advances in understanding listed above were incorporated and accounted for in the recent KiDS cosmic shear analysis \citep{hildebrandt/etal:2016}.  The CFHTLenS tomographic cosmological analysis was also revisited by \citet{joudaki/etal:2016} in order to include a full redshift error analysis, as discussed in more detail in Section~\ref{sec:photoz}.  Efforts are now underway to fully re-analyse CFHTLenS using the advanced KiDS analysis pipeline with revised measurements and calibrations for the shear and photometric redshifts.  

