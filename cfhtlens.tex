%\subsubsection{CFHTLenS updates and advances since 2014}

The Canada-France-Hawaii Telescope Lensing Survey (CFHTLenS) represented a
major step forward for the field of weak gravitational lensing, in terms of
improved accuracy in data reduction \citep{CFHTLenS-data}, the implementation
of PSF-Gaussianised matched multi-band photometry
\citep{CFHTLenS-photoz}, cross-correlation clustering analysis between
photometric redshift slices to verify tomographic redshift distributions
\citep{CFHTLenS-2pt-tomo}, accurate calibrated shape measurements
\citep{CFHTLenS-shapes} and a full suite of informative systematic tests to
select a clean data set \citep{CFHTLenS-sys}. Since the public release
of this survey in 2013, the community has continued to scrutinise and advance
our understanding of CFHTLenS by identifying a number of areas where analyses
could improve:
%
\begin{itemize}
%
 \item{\citet{2016MNRAS.463.3737C} identified significant biases in the tomographic
photometric redshift distributions using a more effective clustering analysis,
in comparison to \citet{CFHTLenS-2pt-tomo}, by incorporating newly overlapping
spectroscopic data from the Sloan Digital Sky Survey.  The conclusion of this
work was that any re-analysis of CFHTLenS should include systematic error terms
to account for bias and scatter, with a prediction that accounting for these
biases would {\it reduce}\/ the recovered amplitude of $\sigma_8$ by $\sim
4$\%. Additional new techniques to calibrate the redshift distribution of tomographic
bins was introduced recently in \cite{KiDS-450}.}
%
\item{The CFHTLenS tomographic cosmological analysis was then revisited by
\citet{joudaki/etal:2016} in order to include a full redshift error analysis
based on the results from \citet{2016MNRAS.463.3737C}.  The impact of
correcting for these biases, including their associated errors, served to
reduce the overall constraining power of the survey and hence also the tension
between CFHTLenS and CMB constraints.}
%
 \item{\cite{asgari/etal:2017} used the stringent COSEBI statistic
\citep{COSEBIs} to identify significant non-lensing B-mode distortions when the
CFHTLenS data was split into tomographic slices.}
%
\item{\citet{2015MNRAS.454.3500K} showed that the CFHTLenS shear calibration
corrections derived in \citet{CFHTLenS-shapes} were underestimated as a result
of an imperfect match between the galaxy populations in the data and image
simulations.}
%
\item{\citet{2016arXiv160605337F} demonstrated that the CFHTLenS data would
have been subject to a weight bias that favours galaxies that are more
intrinsically oriented with the point-spread function.  They also showed that
the impact of calibration selection biases, that were not considered in
\citet{CFHTLenS-shapes}, would have lead to the over-correction of
multiplicative shear bias in the CFHTLenS analyses, by a few percent.}
%
\item{\citet{joudaki/etal:2016} updated the CFHTLenS covariance matrices using
larger-box numerical simulations that were less subject to the lack of power on
large scales. A complementary accurate estimate of the covariance matrix using
analytical methods will be published soon (Joachimi et al.~in prep.)}
%\
\item{\cite{2012ApJ...761..152T} provided a more accurate non-linear power
spectrum correction than that used in the original CFHTLenS analyses, and the
halo model from \cite{2015MNRAS.454.1958M} allowed for simultaneous modelling
of baryonic modifications to the non-linear power spectrum.} 
%
\end{itemize}
%
All these advances in our understanding were incorporated and accounted for in
the recent KiDS cosmic shear analysis \citep{KiDS-450} which reports a $2.3
\sigma$ tension with Planck.  Efforts are now underway to fully re-analyse
CFHTLenS using the advanced KiDS analysis pipeline with revised shape
measurements and calibrations for the shear and photometric redshifts. Until
this analysis is complete we note that these known shortcomings with the
original CFHTLenS results impact in different ways the cosmological conclusions
that one can draw. As CFHTLenS has similar statistical power
to current weak lensing surveys, however, it nevertheless provides a very
useful testbed with which to demonstrate the impact of adopting different
approximations when constraining cosmological parameters.

%\subsubsection{Cosmological analysis setup}

In this work, we focus on the weak-lensing power spectrum projection, and
assess the impact of various approximations on cosmological constraints from
CFHTLenS. For consistency with the original analysis
\citep{CFHTLenS-2pt-notomo}, we adopt the same priors and non-linear power
spectrum corrections from \cite{2003MNRAS.341.1311S}.

We re-analyse the 2D CFHTLenS measurement of the two-point shear correlation
function $\xi_\pm(\theta)$ from \cite{CFHTLenS-2pt-notomo}, defined in
equation~(\ref{eqn:xiGG}). As in \cite{CFHTLenS-2pt-notomo} we fit both
components $\xi_+$ and $\xi_-$ between angular scales $\theta = 0.8$ and $350$
arc minutes, and use a $N$-body simulation estimate of the non-Gaussian
covariance including the cross-covariance between both components. Bayesian
Population Monte-Carlo parameter sampling is performed using the publicly
available software
\textsc{CosmoPMC}\footnote{\texttt{http://www.cosmostat.org/software/cosmopmc}}
\citep{WK09,KWR10}. The cosmological modelling part includes the various
lensing projections, calculated using the software library
\textsc{nicaea}\footnote{\texttt{http://www.cosmostat.org/software/nicaea}}.

%%%%%%%%%%%%%%%%%%%%%%%%%%%%%%%%%%%%%%%%%%%%%%%%%%%%%%
%\subsubsection{Cosmological parameter results}
%\label{ref:cosmo_results}
%%%%%%%%%%%%%%%%%%%%%%%%%%%%%%%%%%%%%%%%%%%%%%%%%%%%%%


For a first-order standard Limber flat-sky approximation (L1Fl) we find
$\sigma_8 (\Omega_{\rm m}/0.27)^{0.6} =0.787^{+0.031}_{-0.033}$, the same
result that was published in \cite{CFHTLenS-2pt-notomo}. Using the second-order
extended Limber flat-sky hybrid approximation (ExtL2FlHyb) results in $\sigma_8
(\Omega_{\rm m}/0.27)^{0.6} = 0.788 \pm 0.032$, a negligible change of the
amplitude that is well within the Monte-Carlo sampling noise. The largest
difference is measured with the \forref{depreciated} case ExtL1Fl, for which the
recovered amplitude is larger by $16\%$ of the statistical error. These
negligible changes to the error bars were to be expected owing to the high
level of statistical noise and cosmic variance in comparison to the low-level
impact of the various approximations shown in Fig.~\ref{fig:Cl_cases}.

Table~\ref{tab:CFHTLenS_Sigma8} lists the mean and 68\% credible interval for
$\sigma_8 \Omega_{\rm m}^{0.6}$ for the various approximations to the
lensing power spectrum projections listed in Table~\ref{tab:cases}. Note again
that these values do not represent the state-of-the-art cosmological results,
since many of the above listed analysis advancements made since 2013 have not
been taken into account. As an example of a significant effect, when using the
revised non-linear power spectrum of \cite{2012ApJ...761..152T} in place of
\cite{2003MNRAS.341.1311S}, there is a decrease of $0.6 \sigma$ with $\sigma_8
(\Omega_{\rm m}/0.27)^{0.6} = 0.768^{+0.029}_{-0.031}$\forref{, using L1Fl}.

Considering the cosmological constraints from tomographic Kilo-Degree Survey
(KiDS), we conclude that these are robust to flat-sky and Limber
approximations. The case ExtL1FlHyb that was used for the analysis of KiDS data
in \citet{KiDS-450} and \cite{joudaki/etal:2017} introduces errors that are
more than an order of magnitude lower than the cosmic variance for that survey,
and thus this approximation has a negligible impact on the cosmological
parameters.



\renewcommand{\baselinestretch}{1.5}
\begin{table}
\begin{centering}
  
  \caption{\label{tab:CFHTLenS_Sigma8}Mean and 68\% confidence interval for 
  $\sigma_8 (\Omega_{\rm m}/0.27)^{0.6}$ and $\sigma_8 (\Omega_{\rm m}/0.3)^{0.6}$
  for various approximations to the lensing
  power spectrum projections listed in Table~\ref{tab:cases}.}

  \begin{tabular}{lcc} \hline
  ID         & $\sigma_8 (\Omega_{\rm m}/0.27)^{0.6}$ & $\sigma_8 (\Omega_{\rm m}/0.3)^{0.6}$ \\ \hline
  L1Fl       & $0.787^{+0.031}_{-0.033}$ & $0.739^{+0.029}_{-0.031}$ \\
  ExtL1Fl    & $0.792 \pm 0.032$ & $0.744 \pm 0.030$ \\
  ExtL1FlHyb & $0.788^{+0.031}_{-0.033}$ & $0.740^{+0.029}_{-0.031}$ \\
  ExtL2FlHyb & $0.788^{+0.031}_{-0.033}$ & $0.740^{+0.029}_{-0.031}$ \\
  ExtL2Sph   & $0.789^{+0.031}_{-0.032}$ & $0.740^{+0.029}_{-0.030}$ \\ \hline
  \end{tabular}

\end{centering}
\end{table}
\renewcommand{\baselinestretch}{1}


